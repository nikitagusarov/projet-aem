\documentclass[11pt, a4paper]{article}

\usepackage[utf8]{inputenc}
\usepackage[french]{babel}
\AddThinSpaceBeforeFootnotes
\FrenchFootnotes
\usepackage[T1]{fontenc}
\usepackage{amsmath}
\usepackage{booktabs}
\usepackage{longtable}
\usepackage{amsfonts}
\usepackage{amssymb, amsthm}
\usepackage{supertabular}
\usepackage[dvipsnames]{xcolor}
\usepackage{geometry}
\usepackage{array}
\setlength{\parindent}{0.1em}
%\setcellgapes{0.1pt}
%\makegapedcells
\newcolumntype{R}[1]{>{\raggedleft\arraybackslash }b{#1}}
\newcolumntype{L}[1]{>{\raggedright\arraybackslash}b{#1}}
\newcolumntype{C}[1]{>{\centering\arraybackslash}b{#1}}
\geometry{hmargin=2.5cm, vmargin=2.5cm}
%\DeclareUnicodeCharacter{00A0}{ }
\setlength{\parskip}{.5em}
\usepackage{comment}
\usepackage[graphicx]{realboxes}
\usepackage[section]{placeins}
\usepackage{adjustbox}
\usepackage{etex}
\usepackage[hypertexnames=false]{hyperref}
\hypersetup{colorlinks = true, linkcolor = Black, urlcolor = Blue, citecolor = Black}
\usepackage{tabu}
\usepackage{pgf, tikz}
\usetikzlibrary{positioning,chains,fit,shapes,calc}
\usetikzlibrary{arrows}
\usepackage{bm}
%\usepackage{lscape}
\usepackage{pdflscape}
\usepackage[splitrule]{footmisc}
\usepackage[nottoc, notlof, notlot]{tocbibind}
\usepackage[clockwise]{rotating}
\renewcommand{\baselinestretch}{1}
%\renewcommand{\thesection}{\Roman{section}}
\usepackage{titlesec}
\titlespacing\section{0pt}{12pt plus 4pt minus 2pt}{0pt plus 2pt minus 2pt}
\titlespacing\subsection{0pt}{12pt plus 4pt minus 2pt}{0pt plus 2pt minus 2pt}
\titlespacing\subsubsection{0pt}{12pt plus 4pt minus 2pt}{0pt plus 2pt minus 2pt}
\usepackage{cite}
\usepackage{easytable}
\usepackage{caption}
\usepackage{lipsum}
\usepackage{alltt}
\usepackage{graphicx}
\usepackage[authoryear, round]{natbib}
%\usepackage{fontspec}
%\usepackage[french]{babel}
%\usepackage{fontspec}
%\usepackage{xunicode}
%\usepackage{xltxtra}
%\setmainfont{Calibri}
\pagenumbering{arabic}
\addto\captionsfrench{\def\tablename{Tableau}}
\DeclareMathOperator*{\argmin}{argmin}
\DeclareMathOperator*{\argmax}{argmax}
%\usepackage{usebib}
%\newcommand{\printarticle}[1]{\citeauthor{#1}, ``\usebibentry{#1}{title}''}
%\bibinput{biblio}
%\usepackage{bibtex}
\bibliographystyle{plainnat}

\begin{document}

\begin{center}
    \Large\textbf{État d'avancement}
    \par
    \large\textit{A. Blanc, N. Gusarov, S. Picon}
\end{center}

% Merci de faire une note avec votre état d'avancement sur la réalisation de votre projet empirique.
% Afin de rendre ce rendu bénéfique : merci d'identifier précisément les points de blocage.
% Je vous demande de me renvoyer ces éléments sur Modèle avant le 5 novembre à 17h00 que je puisse les consulter avant notre cours.

\section{Introduction}
La France est l’un des principaux producteurs et vendeurs de vin dans le monde. En effet, la France représente 10\% de la surface de vigne dans le monde. 
La surface de vigne française se répartit dans 65 des 95 départements de la métropole. 
En France, il y a plus de 750000 hectares de vignes qui sont exploitées en 2018. 
Ainsi, en France, une exploitation agricole sur cinq est une exploitation viticole. Cela représente 85000 exploitations. 
La production de vins en France, représentait 4,6 milliard de litres. 
Cela représentait plus de 17\% de la production totale de vin. 
En volume de production la France se place donc en deuxième position derrière  le volume de production de l’Italie. 
3\% de la surface agricole est consacrée à la production de vin. 
Néanmoins, le vin représente 15\% de la production agricole en valeur. 
Du côté du consommateur, la France est le deuxième pays consommateur de vin derrière les Etats Unis. 
En effet, la consommation de vin en France représentait plus de 3,5 milliards de bouteille, en 2018. 
Néanmoins, on remarque une baisse de la consommation Française depuis une trentaine d’année.
%%% cette partie était verte dans le document
La plupart des bouteilles achetées sont achetées dans la grande distribution. 
Néanmoins, dans un souci de simplicité nous estimerons que les consommateurs achètent leurs bouteilles directement auprès du viticulteur. 
Donc nous supprimerons tous les intermédiaires entre le producteur et le marché final. 
Dans le contexte actuel de forte mondialisation, les vins traversent aussi les frontières. 
Ainsi, la France est le premier exportateur de vin.  
Là aussi, on ne prend pas en compte le marché extérieur et on se limitera uniquement au marché intérieur. 
En effet, nous disposons seulement de données sur le marché intérieur. 
%%% 
En 2018,  la consommation mondiale de vin était de 32 milliards de bouteilles. 
La France consomme 3,7 milliards de bouteilles.
\par
Dans cette étude, nous avons aussi décidé de nous concentrer sur le marché des vins sans indication géographique pour minimiser l’hétérogénéité qui existe entre les vins de différents autres labels. 
Ces vins ont vu leurs transactions augmenter en volume pour toutes les couleurs. 
Ainsi, on remarque que pour les vins rouges les transactions ont augmenté de 10\%, pour les rosées la hausse représentait 52\%, pour les vins blancs les volumes de transactions ont presque été doublé. 
Néanmoins on remarque une baisse des cours des vins sans indication géographique. 
En effet, on remarque que les prix moyens pour les vins rouges et rosées sans indication géographique baisse de 3\%. 
Le prix moyen des vins blancs baisse quand à eux de 12\%, pour la campagne 2019/2020. 
Sur les deux mois de campagne, les échanges de Vin sans indication géographique est de 142 milliers d’hectolitres. 
Cela correspond à une hausse de 39\% par rapport à la campagne précédente. 
Les ventes représentent 92 milliers d’hectolitres. 
La tendance sur le marché des vins sans indication géographique s’explique par une forte hausse des vins blancs. 
En effet, ceux-ci connaissent une hausse de près de 28 milliers d’hectolitres, soit une hausse de 232\% vis-à-vis de la campagne de 2018-2019. Les vins rosés connaissent également une hausse. 
Néanmoins, celle-ci reste modeste puisque les ventes augmentaient de 61\% par rapport à la campagne 2018/2019. 
Néanmoins, les ventes de vins rouges ont légèrement baissé. 
Le cours des Vins sans indication géographique baisse par rapport à la campagne précédente. 
\par
Lors de la campagne 2018/2019, les ventes de vins en grande distribution sont en baisse. 
Cela peut s’expliquer par une hausse des prix moyens. 
Les ventes de vins représentent 8,7 millions d’hectolitres et un chiffre d’affaires de 4,1 milliards d’euros avec un prix moyen de 4,73 euro/litre. 
La baisse de la consommation de vins rouges s’aggrave avec une baisse de 8\% par rapport à la campagne de 2017/2018. 
Les vins blancs connaissent aussi une faible baisse de 1,2\% en volume par rapport à la consommation de la campagne précédente. 
Pour finir, les ventes de vins rosés ont baissé lors de la campagne 2018/2019. 
En effet, on enregistre une baisse de 3,9\% en volume par rapport à la campagne 2017/2018. 
La consommation de vin sans indication géographique est de 6\% en volume contre 3\% en valeur. 
Les ventes de vins sans indications géographiques sont en légère hausse dans la campagne 2018/2019 par rapport à la campagne 2017/2018. 
\par
Les phytosanitaires sont très utilisés dans les cultures comme la viticulture. 
Il s’agit donc d’un intrant important pour la production de vin. Ainsi, la viticulture utilisait 15\% de produit phytosanitaire. 
La pression sanitaire varie selon les productions. 
Elle est forte en viticulture. 
De la même façon, la pression phytosanitaire varie selon les régions. 
Ainsi, pour la vigne l’IFT varie de 7 en Provence à 22 en Champagne.

\section{Question économique traitée}
Dans notre travail nous étudions les impacts des fluctuations du prix des vins de table (les vins simples) sur la demande des pesticides par des agriculteurs français.

% \section{Références et synthèses}
% % référence et synthèse des 2/3 articles fondateurs
% Pour le moment nous avons étudié et synthétisé 14 articles sur le marché du vin, ainsi bien que 2 articles sur le traitement économétrique des systèmes d'équations simultanés. 
% % \par
% % Parmi les articles traitant la viticulture et le marché du vin, nous avons les travaux de :
% % \begin{itemize}
% %     \item \citet*{rebelo2019}
% %     \item \citet*{anderson2018s}
% %     \item \citet*{bazen2018}
% %     \item \citet*{garcia2016}
% %     \item \citet*{cembalo2014}
% %     \item \citet*{outreville2010}
% %     \item \citet*{benfratello2009}
% %     \item \citet*{davis2008}
% %     \item \citet*{christophe2008}
% %     \item \citet*{costanigro2007}
% %     \item \citet*{steiner2004}
% %     \item \citet*{soler2002}
% %     \item \citet*{giraud2001}
% %     \item \citet*{gergaud1998}
% %     \item \citet*{labys1976}
% %     \item \citet*{daillens1962}
% % \end{itemize}
% % Les travaux théoriques et méthodologiques sont des extraits du manuel de Wooldridge :
% % \begin{itemize}
% %     \item \citet*{wooldridge2016introductory}
% %     \item \citet*{wooldridgepanel}
% % \end{itemize}
% % \par
% % Maintenant, nous allons donner un aperçu des articles les plus pertinents dans notre cas d'étude.

% \subsection{"Drinking cheaply: the demand for basic wine in Italy"}
% Un travail de Luigi Cembalo, Francesco Caracciolo et Eugenio Pomarici, publié en 2014.
% Ce recherche nous servira pour la base, car il traite la plus part des problèmes dont on rencontre dans cet étude. 
% \par
% Cet article détermine s'il existence un éventuel degré d’hétérogénéité dans les vins non premium et essaye le mesurer.
% Cela est fait par calcul de l’élasticité, les relations entre les catégories de vins agrégées avec des critères allant au-delà du prix (producteur, type, couleur, etc). 
% Un système de demande (modèle QUAIDS) a été estimé à l'aide d'un panel statistiquement représentatif de 6 773 ménages italiens, afin de déterminer dans quelle mesure une éventuelle substitution se produit dans la consommation domestique de vins de base (qui constitue le principal canal de distribution des vins bon marché en Italie).
% \par
% Les résultats les plus saillants sont la faible élasticité du vin en carton du leader du marché (ML) et la grande élasticité du vin non-marque en carton (OC), ce qui indique sur la présence d'un degré d'hétérogénéité. 
% Sur la base des résultats, les auteurs affirment que, même dans le segment des vins plus économiques, la marque est un instrument efficace de diversification.
% \par
% Nous dans notre analyse, supposons, que ces effets d'hétérogénéité sont similaires pour des groupes des marques, agrégées par région.

% \subsection{"French Wines on the decline ? Econometric evidence from Britain"}
% Cet analyse effectué par Bodo Steiner en 2002 contribue à la littérature existante des prix hédoniques sur le marché du vin en UK. 
% \par
% Les auteurs développent l’approche des variables nominales % mise au point par Kennedy (1986) et Oczkowski (1994) 
% pour obtenir une contribution distincte et comparable de chaque attribut à la variation des prix. 
% Dans leurs étude, % contrairement à Combris et al. (1997), 
% ils s'appuis pas sur des caractéristiques sensorielles, mais sur des attributs que les consommateurs peuvent observer via l'étiquette (cépage, millésime, etc.). 
% De plus, contrairement aux études hédoniques précédentes sur le vin, les chercheurs n'utilisent pas les prix de détail recommandés, mais les prix de détail réels.
% \par
% L’approche économétrique utilisé est un estimateur des moindres carrés avec la correction pour l’hétéroscédasticité. 
% \par
% Les estimations suggèrent que les vins de cépages français, tels qu’ils sont apparus dans les points de vente au détail en 1994, n’avaient pas obtenu une évaluation suffisamment positive des acteurs du marché britannique. 
% Les auteurs constatent la diminution globale du rôle des vins français en Grande-Bretagne.

% \subsection{"Les facteurs déterminants les prix du vin"}
% Un travail de J. François Outreville fait en 2009.
% \par
% L'auteur identifie les facteurs les facteurs les plus fréquemment utilisé dans la littérature portant sur l'analyse des vins et du marché du vin.
% \begin{center}
% \begin{tabular}{l|l|l}
%     \hline
%     \hline
%     Facteurs géo-viticoles & Facteurs temporels & Coûts de production \\
%     \hline
%     ~~~~ Climat & ~~~~Age & ~~~~Coûts fixes et variables \\
%     ~~~~ Sol & ~~~~Millésime & ~~~~Type de vin \\
%     ~~~~ Région & & ~~~~Rendement \\
%     ~~~~ Cépage & & \\
%     \hline
% \end{tabular}
% \par
% \begin{tabular}{l|l}
%     \hline
%     \hline
%     Information & Offre et demande \\
%     \hline
%     ~~~~Etiquette & ~~~~Rareté \\
%     ~~~~Appellation & ~~~~Culture \\
%     ~~~~Réputation & \\
%     & \\
%     \hline
% \end{tabular}
% \end{center}

% \subsection{"Pesticide regulation : the case of French wine"}
% C'est article est un "\textit{working paper}" rédigé par Deola Cristophe et Fleckinger Pierre en 2008.
% \par
% Les auteurs identifient que le régime d’appelation favorise l’utilisation des pesticides tout en limitant la quantité de raisins utilisée dans la production de vin par rapport à la situation non réglementée. 
% \par
% Consécutivement, les rendements des plantation viticoles sont limitées par certaines réglementation professionnelles si bien que l’utilisation des pesticides est peu sensible à un faible niveau d’imposition car incitation à administrer une plus forte dose de pesticides sur les cultures pour permettre aux producteurs de garantir une bonne qualité du vin produit et donc la perception de revenus suffisants pour que leur exploitation perpétue dans le temps. 
% Une réglementation économique à elle seule n’est pas suffisante pour limiter les dommages environnementaux commis par les pesticides.
% \par
% Les résultats de cet article nous permet de supposer que une faible élasticité-prix de la demande des pesticides nous permet ignorer les relations agriculteur-producteur du marché des pesticides.
% Nous supposons, que la demande des pesticides ne depend que des besoins des agriculteurs.

\section{Modèle économique}
% Cembalo2014
Dans le commerce du vin, il est courant de diviser les vins en deux grandes classes en fonction de leurs prix \cite{cembalo2014} : 
\begin{itemize}
    \item les vins de qualité inférieure, les moins chers avec les caractéristiques de qualité de base ;
    \item les vins de qualité supérieure plus chers, dotés de caractéristiques qualitatives complexes et d'une image de grande valeur.
\end{itemize} % (Anderson et Nelgen 2011).
\par
De plus, pour les vins français, selon \citet{steiner2004}, le système européen de classification des "\textit{vins de qualité produits dans certaines régions}" (VQPRD) contient à la fois des vins AOC et des "\textit{vins de haute qualité provenant d'un vignoble régional agréé}" (VDQS). 
Les vins de cépage appartiennent à la catégorie des vins autres que VQPRD, qui comprend les \textbf{vins de table} et les \textbf{vins de pays}.
\par
En tenant compte cet information, nous utilisons la méthodologie du ministère d'agriculture et divisons le marché en deux parties :
\begin{itemize}
    \item La gamme haute (les vins IGP, vendus dans des magasins spécifiques) ;
    \item La gamme basse (les vins non IGP, vendus en grands surfaces).
\end{itemize}
\par
La première partie est soumise à des règlements spécifiques : limitations des quantités produites, origine contrôlé, un caractère de la demande spécifique. 
La deuxième, c'est-à-dire le marché des vins moins chers, est aussi complexe. Les produits classés dans cette catégorie sont susceptibles d'avoir un certain degré d'hétérogénéité, comme cela a été montré par \citet{cembalo2014}.
\par
Dans notre étude, nous traitons uniquement les vins simples (non IGP). 
La situation sur ce marché est sensée influencer l'utilisation des pesticides, car les volumes de productions sont plus significatives que pour le marché des vins IGP.
\par 
Suivant le raisonnement des chercheurs \cite{cembalo2014}, dans une catégorie de vin avec une fourchette de prix étroite, il existe une homogénéité presque parfaite due à des vins ayant des attributs intrinsèques simples, une complexité de qualité médiocre et donc une différenciation peu marquée.
Nous ignorons les interactions internationales. 
Cela nous permet d'analyser le marché par département est non par des marques/produits.
\par
Comme proposé dans la littérature, notre étude sur les vins non coûteux (non IGP) est effectué au niveau du pays \cite{cembalo2014} pour deux raisons :
\begin{itemize}
    \item Les prix de vente moyens des marchés sont diffèrent en raison des droits de douane à l'importation et des taxes à la consommation différents % (Anderson et Nelgen 2011);
    \item La perception des produits de consommation varie d'un pays à l'autre % (Makela et al. 2006).
\end{itemize}
Quand aux exportations et les importations, n'ayant pas la possibilité contrôler le montant des vins non IGP exportés/importés, nous laissons ces effets au terme d'erreur. 
\par
Pour conclure, nos suppositions au niveau du marché des vins sont les suivantes :
\begin{itemize}
    \item La demande pour les vins simples est unique pour toute la France. On n'observe pas les quantités consommés par départements, mais pour tout le pays, avec un prix unique. 
    \item La production du vin varie par département, suite à des différences climatologiques.
    \item On n'observe que l'équilibre sur le marché au niveau du pays (la quantité demandé est égale à la quantité offerte par l'ensemble des régions).
\end{itemize}
\par
En ce qui concerne les pesticides, nous supposons que :
\begin{itemize}
    \item La demande des pesticides est inélastique au prix, ce qui nous permet d'exclure la partie de l'offre des pesticides du notre analyse. La quantité de pesticides utilisée demande seulement des intentions et des besoins des agriculteurs. 
\end{itemize}
\par
En formalisant notre modèle théorique, nous posons, que la demande de vin a la forme suivante :
\begin{equation}
    Q_d = \alpha_d + \beta_d P_d + \gamma_d Z 
\end{equation}
Avec $Z$ étant l'ensemble des variables ayant une influence sur la demande du vin, dans le cas le plus simple nous n'utilisons que les revenus (c'est une des variables les plus utilisées dans des études empiriques sur le marché du vin).
\par
L'offre totale pour toute la France est donnée par l'équation suivante : 
\begin{equation}
    Q_o = \sum_{i = 1}^{N} q_i
\end{equation}
Ou $i \in \{1, ..., N\}$ sont des départements, chacun ayant sa propre fonction de production et d'offre unique : 
\begin{equation}
    q_i = a_i + b_i P_o + c_i X
\end{equation}
Avec $X$ étant un vecteur des variables explicatives influençant la production (dans le cas le plus simple nous ne prenons en compte que les quantités des pesticides utilisées).
Nous pouvons réécrire l'équation de l'offre sous la forme :
\begin{equation}
    Q_o = \sum_{i = 1}^{N} (a_i + b_i P_o + c_i X) = \sum_{i = 1}^{N} a_i + \sum_{i = 1}^{N} b_i P_o + \sum_{i = 1}^{N} c_i X
\end{equation}
Nous obtenons enfin un système de $N + 2$ équations : 
\begin{align*}
    Q_d & = \alpha_d + \beta_d P_d + \gamma_d Z \\
    Q_o & = \sum_{i = 1}^{N} q_i \\
    q_{1,t} & = a_1 + b_1 P_o + c_1 X \\ 
    \vdots \\ 
    q_{N,t} & = a_N + b_N P_o + c_N X \\
\end{align*}
Quand même, parce que nous pouvons supposer une presence des contraintes au niveau des données, nous devrions prévoir des modifications possibles pour notre modèle.
Les contraintes principales sont au niveau du manque des données au niveau des années, c'est-à-dire que nous risquons d'avoir une trés faibles variation intra-annuelle des prix et de revenus pour pouvoir identifier les coefficient associés par un passage à l'équation structurelle.
\par
Une de ces modification possibles est l'introduction d'une contrainte supplementaire au niveau de la demande sur le vin de table.
Afin de pouvoir identifier les effets de toutes les variables par un système AIDS, nous pouvons supposer, que tout le vin produit dans un département est consommé dans le même department.
C'est une hypothèse forte, qui nous éloigne de la réalité, parce que de cette façon nous ignorons plusieurs effets pervers, tels que :
\begin{itemize}
  \item La structure du marché interne de la France ;
  \item La mobilité de la production entre les differents départements ;
  \item L'export du vin ;
  \item La consommation des vins importés.
\end{itemize}
% Ici il faut ajouter des liens avec des autres études du marché ...
\par
Nous pouvons tout de méme ignorer ces effets, car nous visons à estimer les effets moyens pour tous les départements. 
De cette façon lors d'aggregation des effets au niveau national nous allons mitiger les biais possibles.
\par
Alors,nous pouvons réécrire notre système d'equations sous la forme suivante :
\begin{align*}
  qd_1 & = \alpha_{1} + \beta P_{1,d} + \gamma_{1} Z_{1} \\
  \vdots \\ 
  qd_N & = \alpha_{N} + \beta P_{N,d} + \gamma_{N} Z_{N} \\
  qo_1 & = a_1 + b P_{1,o} + c_1 X_{1} \\ 
  \vdots \\ 
  qo_N & = a_N + b P_{N,o} + c_N X_{N} \\
\end{align*}
Il faut specifier, que nous supposons les effets de prix sont identiques pour tous les département en moyenne, tandis que nous laissons quand même les effets des autres variables dépéndantes (ex : le revenu et les pésticides) de varier par département.

\section{Les données}
Nous avons utilisé les bases des données suivantes pour notre analyse :
\begin{itemize}
    \item Les données de ventes de pesticides par département : Institut National de l’Environnement Industriel et des Risques (INERIS)
    \item Les données sur les prix du vin (France Agrimer)
    \item Les données sur la population (INSEE)
    \item Les données sur la production de vin (service statistique du ministère des Finances)
\end{itemize}
\par
Au niveau des pesticides, on va s’intéresser plus particulièrement aux quantités de produits vendus par département entre 2009 et 2017 utilisés principalement sur les cultures viticoles. 
Il faut faire preuve de vigilance sur le conditionnement des produits qui n’est pas exprimé dans la même unité au sein de cette base : en litres ou en kilos.
Quand même nous allons étudier l'impact de la masse totale des pésticides utilisés.
Pour pouvoir le faire, nous créons un indice qui permet de prendre en compte les évolutions des differents types des produits à la fois.
Nous créons un indice simple :
\begin{equation*}
  P = \frac{\sum_j p_{j, t} q_{j, t}}{\sum_j p_{j, 0} q_{j, 0}}
\end{equation*}
Avec $j$ désignant le produit $j$.
\par
En ce qui concerne les données sur le prix du vin, on s’intéresse principalement au prix moyen des vins rouge- rosés et blancs sans IG (Indication Géographique) sur la période 2009-2017. 
Ces prix sont déflatés par l’indice des prix à la consommation (base 100 en 2014). 
On ne considère ici que le prix moyen déflaté au niveau national.
Dans le deuxième modèle nous avons besoin de créer artificiellement un estimateur qui va varier par département.
Dans ce but nous créons l'indice de prix du vin de table départementale, calculé de façon suivante :
\begin{equation*}
  P = \frac{p_{rouge, t} q_{rouge, t} + p_{blanc, t} q_{blanc, t}}{p_{rouge, 0} q_{rouge, 0} + p_{blanc, 0} q_{blanc, 0}}
\end{equation*}
Avec $t$ étant l'anée au temps $t$.
\par
Au niveau des données sur la population, la variable qui nous intéresse ici est relative au niveau de revenu, exprimée au niveau départemental (laquelle, si besoin nous pourrions facilement aggréger au niceau national). 
Plus précisément, on va utiliser le revenu médian par département.
Il est aussi déflatée de l’indice des prix à la consommation (base 100 en 2014).
\par
Enfin, nous avons exploité les variables suivantes au niveau de la production de vin : la surface totale de culture viticole en hectares, la surface utilisé pour les vins non IG, la quantité produite de vins rouges-rosés et blancs sans IG en hectolitres, pour chaque département et sur la période 2009-2017.

\subsection{Les données mobilisées}
% les étapes de la construction de votre BDD ainsi que les différents appariements sont à mentionner.
Nous avons commencé par chercher à fusionner les bases de données relatives à la production de vin puis aux quantités de pesticides vendues. 
Dans cette optique, nous allons utiliser deux clefs de fusion : le département et l’année. 
Néanmoins, au niveau des prix et des revenus, nous avons décidé de nous placer à l’échelon national par manque de données disponibles suffisamment détaillées pour caractériser chaque département. 
Nous avons donc concaténer ces deux bases en se basant uniquement sur l’année comme clef de fusion.
\par
En ce qui concerne les points de blockage, nous allons probablement rencontrer un problème en ce qui concerne l’identification des variables dépendantes qui sont les prix et les quantités. 
Ce problème d’identification provient potentiellement d’un manque de variabilité dans notre jeu de donnée lié au choix d’un niveau de prix moyen au niveau  national  qui entraînerait une estimation biaisée de l’influence du prix sur les quantités demandées de pesticides. 
En effet, il est très probable que cet impact change selon le département et donc notre estimation ne tienne pas assez compte des disparités nationales.

\subsection{Dictionnaire des variables.}
\FloatBarrier
\begin{table}[!htbp]
  \centering
\begin{tabular}{c|l}
  \hline
  Variable & Description \\
  \hline
année & année \\
dep & département \\
s\_nig & superficie de vigne sans indication géographique en hectare \\
s\_total & superficie de vigne totale en hectare \\
q\_blanc & quantité de vins blancs produits en hectolitre \\
q\_rouge & quantité de vins rouges produits en hectolitre \\
q\_total & quantité totale de vins produits en hectolitre \\
p\_blanc & prix moyens des vins blancs sans indication géographique en euros par hectolitre déflatés \\
p\_rouge & prix moyens des vins rouges sans indication géographique en euros par hectolitre déflatés \\
revenu & revenu disponible brut des ménages français déflatés \\
qk\_prod & quantité de produits de pesticides achetés en kilogrammes \\
ql\_prod & quantité de produits de pesticides achetés en litre \\
\hline
\end{tabular}
\caption{Ditionnaire des varibales}
\end{table}
\FloatBarrier
Les données sont transformé par la fonction du R suivante :
\begin{alltt}
  datai = datax %>%
    arrange(ndep) %>%
    mutate(si = log(s_vin_simple + 0.001), 
        qi = log(q_blanc + q_rouge + 0.001), 
        ipi = log(IP),
        ri = log(revenu.déflaté),
        iki = log(IQK),
        t = as.integer(as.factor(annee))) %>%
    dplyr::select(ndep, qi, ipi, si, ri, iki, t)
\end{alltt}
\par
Alors, les variables qu'on va inclure dans notre équation sont :
\par
\FloatBarrier
\begin{table}[!htbp]
  \centering
  \begin{tabular}{c|l}
    \hline
    Variable & Description \\
    \hline
  si & superficie de vigne sans indication géographique en log \\
  qi & quantité de vins blancs produits en log \\
  ipi & indice des prix des vins sans indication géographique en log \\
  ri & revenu disponible brut des ménages français déflatés en log \\
  iki & indice des quantité de produits de pesticides achetés en log \\
  t & la tendance temporelle \\
  \hline
  \end{tabular}  
\caption{Final variables}
\end{table}
\FloatBarrier
Les propriétés de ces données sont suivantes :
\begin{itemize}
  \item Toutes les variables varient par département et par année.
  \item Le période temporelle comprise dans notre échantillon est de 2012 à 2016.
  \item Nous ne considérons que les régions produisant du vin. 
  \item Nous éliminons les effets fixes pour en substrayant les moyennes départamentales.
  \item Données en panel "cylindrées".
  \item Nombre des individus large (69 départements, qui produisent le vin et utilisent des pésticides) et le nombre des périodes pauvre (5 périodes).
\end{itemize}

\subsection{Statistiques descriptives des variables clé}
Maintenant, nous pouvons passer à l'étude des variables clé.

\subsubsection{Les statistiques générales}
\FloatBarrier
\begin{table}[!htbp]
  \centering
\begin{tabular}{l|c}
  \hline
   & Overall (N=345)\\
  \hline
  \textbf{ndep} & \\
  \hline
  ~~~Mean (SD) & 44.580 (26.322)\\
  \hline
  ~~~Range & 1.000 - 89.000\\
  \hline
  \textbf{qi} & \\
  \hline
  ~~~Mean (SD) & -0.000 (0.423)\\
  \hline
  ~~~Range & -1.900 - 0.937\\
  \hline
  \textbf{ipi} & \\
  \hline
  ~~~Mean (SD) & 0.175 (0.568)\\
  \hline
  ~~~Range & -1.654 - 2.921\\
  \hline
  \textbf{si} & \\
  \hline
  ~~~Mean (SD) & -0.000 (0.410)\\
  \hline
  ~~~Range & -1.204 - 2.503\\
  \hline
  \textbf{ri} & \\
  \hline
  ~~~Mean (SD) & 0.000 (0.011)\\
  \hline
  ~~~Range & -0.038 - 0.044\\
  \hline
  \textbf{iki} & \\
  \hline
  ~~~Mean (SD) & 0.170 (0.333)\\
  \hline
  ~~~Range & -1.034 - 1.467\\
  \hline
  \textbf{t} & \\
  \hline
  ~~~Mean (SD) & 3.000 (1.416)\\
  \hline
  ~~~Range & 1.000 - 5.000\\
  \hline
  \end{tabular}
\caption{Statistiques déscriptives}
\end{table}
\FloatBarrier

\subsubsection{Analyse de la correlation}
\FloatBarrier
\begin{table}[!htbp]
  \centering
\begin{tabular}{r|rrrrrr}
  \hline
  \hline
 & qi & ipi & si & ri & iki & t \\ 
  \hline
qi & 1.00 & 0.74 & 0.37 & -0.16 & -0.16 & -0.20 \\ 
  ipi & 0.74 & 1.00 & 0.22 & -0.01 & -0.13 & 0.04 \\ 
  si & 0.37 & 0.22 & 1.00 & -0.17 & -0.13 & -0.31 \\ 
  ri & -0.16 & -0.01 & -0.17 & 1.00 & 0.16 & 0.65 \\ 
  iki & -0.16 & -0.13 & -0.13 & 0.16 & 1.00 & 0.29 \\ 
  t & -0.20 & 0.04 & -0.31 & 0.65 & 0.29 & 1.00 \\ 
   \hline
\end{tabular}
\caption{Correlation}
\end{table}
\FloatBarrier

\section{Modèles économétriques}
L'AIDS et les autres modèles de demande cités dans la littérature ont de nombreuses lacunes qui les rendent impropres pour l'estimation du marché du vin, selon \citet{cembalo2014}. 
Quand même, dans notre étude nous allons utiliser ce modèle là, sous des suppositions restrictives. 
\par
Dans cette étude, nous nous intéressons à l’effet de la quantité de pesticides utilisé sur l’équilibre du marché des vins de table.

\subsection{Modèle numero 1}
Formalisant notre prémiere modèle théorique, nous posons, que la demande agrégé de vin a la forme suivante :
\begin{equation}
    Qd_t = \alpha_d + \beta_d Pd_t + \gamma_d Z_t
\end{equation}
Avec $Z$ étant l'ensemble des variables ayant l'influence sur la demande du vin, dans le cas le plus simple nous n'utilisons que les revenus (c'est une des variables les plus utilisées dans des études empiriques sur le marché du vin).
\par
L'offre agrégé pour toute la France est donnée par l'équation suivante : 
\begin{equation}
    Qo_t = \sum_{i = 1}^{N} q_{i,t}
\end{equation}
Avec :
\begin{itemize}
  \item $Qd$ : la quantité demandée de vin en hectolitre
  \item $Pd$ : le prix du vin sous la forme d'indice
  \item $Z$ : le revenu disponible brut déflaté
\end{itemize}
Ou $i \in \{1, ..., N\}$ sont des régions, chacun ayant sa propre fonction de production et d'offre unique : 
\begin{equation}
    q_{i,t} = a_i + b_i Po_t + c_i X_{i,t}
\end{equation}
Avec $X$ étant un vecteur des variables explicatives influençant la production (dans le cas le plus simple nous ne prenons en compte que les quantités des pesticides utilisées).
Plus précisément :
\begin{itemize}
  \item $q_i$ : la quantité de vin en hectolitres dans chaque département
  \item $Po$ : le prix moyen sous la forme d'indice
  \item $X$ : la quantité de pesticide
  \item $Y$ : la superficie en hectare
\end{itemize}
Nous pouvons réécrire l'équation de l'offre sous la forme :
\begin{equation}
    Qo_{i,t} = \sum_{i = 1}^{N} (a_i + b_i Po_{t} + c_i X_{i,t}) = \sum_{i = 1}^{N} a_i + \sum_{i = 1}^{N} b_i Po_{t} + \sum_{i = 1}^{N} (c_i X_{i,t})
\end{equation}
Nous obtenons enfin un système de $N + 2$ équations : 
\begin{align*}
    Qd_t & = \alpha_d + \beta_d Pd_t + \gamma_d Z_t \\
    Qo_t & = \sum_{i = 1}^{N} q_{i,t} \\
    q_{1,t} & = a_1 + b_1 Po_{t} + c_1 X_{1,t} \\ 
    \vdots \\ 
    q_{N,t} & = a_N + b_N Po_{t} + c_N X_{N,t} \\
\end{align*}
A l'équilibre nous ayons $Po_t = Pd_t = P_t$ et $Qo_t = Qd_t = Q_t$.
\par
Alors, n'ayant les valeurs que pour l'équilibre, nous pouvons réécrire notre modèle comme :
\begin{align*}
  Q_t & = \alpha_d + \beta_d P_t + \gamma_d Z_t + \epsilon_t \\
  Q_t & = \sum_{i = 1}^{N} q_{i,t} \\
  q_{1,t} & = a_1 + b_1 P_{t} + c_1 X_{1,t} + u_{1,t}\\ 
  \vdots \\ 
  q_{N,t} & = a_N + b_N P_{t} + c_N X_{N,t} + u_{N,t}\\
\end{align*}
Ce qui nous donne : 
\begin{equation}
    \alpha_d + \beta_d P_t + \gamma_d Z_t + \epsilon_t = 
        \sum_{i = 1}^{N} a_i + \sum_{i = 1}^{N} b_i P_t + \sum_{i = 1}^{N} c_i X_{i,t} + \sum_{i = 1}^{N} u_{i,t}
\end{equation}
Le problème apparaisse au niveau du terme $\sum_{i = 1}^{N} (c_i X_{i,t})$. Si $cor(c_i X_{i,t}) \neq 0$ on a autant des termes $c_i$ dans notre équation de départ que le nombre des départements étudié $N$. 
C'est à nous obtiendrons une équation structurelle du type :
\begin{equation}
    P_t = \frac{\sum_{i = 1}^{N} a_i - \alpha_d}{\beta_d - \sum_{i = 1}^{N} b_i} + 
        \frac{\sum_{i = 1}^{N} c_i  X_{i,t}}{\beta_d - \sum_{i = 1}^{N} b_i} +
        \frac{-\gamma_d}{\beta_d - \sum_{i = 1}^{N} b_i} Z_t + 
        \frac{\sum_{i = 1}^{N} u_{i,t} - \epsilon_t}{\beta_d - \sum_{i = 1}^{N} b_i}
\end{equation}
Cela nous risque de poser des problèmes lors d'estimation et dérivation des coefficients des équation de départ.
Quand même, si nous posons que $c_i$ n'est pas corrélé avec $X_{i,t}$ et $cor(c_i X_{i,t}) = 0$, nous pouvons supposer que :
\begin{equation}
    \sum_{i = 1}^{N} c_i  X_{i,t} = \frac{1}{N} \sum_{i = 1}^{N} c_i \frac{1}{N} \sum_{i = 1}^{N} X_{i,t}
\end{equation}
Ce qui revienne de l'idée que $E(XY) = E(X)E(Y)$ si $cor(X,Y) \neq 0$.
Dans ce cas, l'équation structurelle s'écrit comme :
\begin{equation}
  P_t = \frac{\sum_{i = 1}^{N} a_i - \alpha_d}{\beta_d - \sum_{i = 1}^{N} b_i} + 
      \frac{\sum_{i = 1}^{N} c_i}{\beta_d - \sum_{i = 1}^{N} b_i} \frac{1}{N} \sum_{i = 1}^{N} X_{i,t} +
      \frac{-\gamma_d}{\beta_d - \sum_{i = 1}^{N} b_i} Z_t + 
      \frac{\sum_{i = 1}^{N} u_{i,t} - \epsilon_t}{\beta_d - \sum_{i = 1}^{N} b_i}
\end{equation}
Ce qu'on peut réécrire comme :
\begin{equation}
  P_t = \pi_1 + 
      \pi_2 \frac{1}{N} \sum_{i = 1}^{N} X_{i,t} +
      \pi_3 Z_t + 
      v_t
\end{equation}
Respectivement on peut dériver équation structurelle pour $Q$ :
\begin{multline}
    Q_t = (\alpha_d + \beta_d \frac{\sum_{i = 1}^{N} a_i - \alpha_d}{\beta_d - \sum_{i = 1}^{N} b_i}) + 
        (\beta_d \frac{\sum_{i = 1}^{N} c_i}{\beta_d - \sum_{i = 1}^{N} b_i}) \frac{1}{N} \sum_{i = 1}^{N} X_{i,t} + \\
        (\gamma_d + \beta_d \frac{-\gamma_d}{\beta_d - \sum_{i = 1}^{N} b_i}) Z_t + 
        (\epsilon_t + \beta_d \frac{\sum_{i = 1}^{N} u_{i,t} - \epsilon_t}{\beta_d - \sum_{i = 1}^{N} b_i})
\end{multline}
Ce qui se réécrit sous forme :
\begin{equation}
  Q_t = \theta_1 + 
      \theta_2 \frac{1}{N} \sum_{i = 1}^{N} X_{i,t} +
      \theta_3 Z_t + 
      w_t
\end{equation}
Le reste est estimé comme :
\begin{multline}
  q_{i,t} = (a_i + b_i \frac{\sum_{i = 1}^{N} a_i - \alpha_d}{\beta_d - \sum_{i = 1}^{N} b_i}) + 
      (c_i + b_i \frac{\sum_{i = 1}^{N} c_i}{\beta_d - \sum_{i = 1}^{N} b_i}) X_{i,t} + \\
      (b_i \frac{-\gamma_d}{\beta_d - \sum_{i = 1}^{N} b_i}) Z_t + 
      (u_{i,t} + b_i \frac{\sum_{i = 1}^{N} u_{i,t} - \epsilon_t}{\beta_d - \sum_{i = 1}^{N} b_i})
\end{multline}
En simplifiant on le réécrit : 
\begin{equation}
  q_{i,t} = \psi_{i,1} + 
      \psi_{i,2} X_{i,t} +
      \psi_{i,3} Z_t + 
      e_{i,t}
\end{equation}
Ce qui avec $i$ le numéro de département, nous donne suffisamment des différences entre les coefficients pour identifier les paramètres des équations de départ. 
\par
Avec les deux premières équations structurelles on obtient les coefficients pour la première équation de départ, qui décrit la demande agrégé :
\begin{align}
  \alpha_d & = \theta_1 - \frac{\pi_1 \theta_2}{\pi_2} \\
  \beta_d & = \frac{\theta_2}{\pi_2} \\
  \gamma_d & = \frac{\theta_2 \pi_3}{\pi_2} - \theta_3
\end{align}
Ainsi bien que pour celle, qui décrit l'offre agrégé :
\begin{align}
  \sum_{i = 1}^{N} a_i & = \theta_1 - \frac{\pi_1 \theta_3}{\pi_3} \\
  \sum_{i = 1}^{N} b_i & = \frac{\theta_3}{\pi_3} \\
  \sum_{i = 1}^{N} c_i & = \theta_2 - \frac{\theta_3 \pi_2}{\pi_3}
\end{align}
Les coefficients uniques pour les régions $a_i$, $b_i$ et $c_i$ sont a identifier séparément avec les estimateurs du reste des équations.
On les obtient d'une manière suivante :
\begin{align}
  a_i & = \psi_{i,1} - \frac{\psi_{i,3} \pi_1}{\pi_3} \\
  b_i & = \frac{\psi_{i,3}}{\pi_3} \\
  c_i & = \psi_{i,2} - \frac{\psi_{i,3} \pi_2}{\pi_3}
\end{align}
\par
En ce qui concerne la variance des estimateurs obtenus, il reste encore à vérifier.

\subsection{Modèle numero 2}
Maintenant passons au deuxième modèle qui apparaisse suite à des problèmes d'identification possibles pour le modèle 1. 
\par
Par ce modèle nous visons à estimer les effets moyenns pour tous les départements. 
De cette façon lors d'aggregation des effets au niveau national nous allons mitiger les biais possibles, liés à la misspecification du modèle.
\par
Nous pouvons réécrire notre système d'equations sous la forme suivante :
\begin{align*}
  qd_{1,t} & = \alpha_{1} + \beta Pd_{1,t} + \gamma_{1} Z_{1,t} + \epsilon_{1,t}  \\
  \vdots \\ 
  qd_{N,t} & = \alpha_{N} + \beta Pd_{N,t} + \gamma_{N} Z_{N,t} + \epsilon_{1,t}  \\
  qo_{1,t} & = a_1 + b Po_{1,t} + c_1 X_{1,t} + u_{1,t} \\ 
  \vdots \\ 
  qo_{N,t} & = a_N + b Po_{N,t} + c_N X_{N,t} + u_{N,t} \\
\end{align*}
Nous posons que l'offre et la demande sont egaux au niveau de département. 
L'offre de département vise à satisfaire la demande interne du même département. 
En termes d'aggregation ex-post des effets estimés, nous sommes sensé de tomber sur l'équilibre au niveau du marché national. 
C'est-à-dire :
\begin{align*}
  qd_{1,t} & = q_{1,t}o \\
  \vdots \\ 
  qd_{N,t} & = q_{N,t}o \\
\end{align*}
Au pint d'équilibre nous avons également l'égalité des prix :
\begin{equation*}
  Po_{1,t} = Pd_{1,t}
\end{equation*}
De cette façon nous obtenons un système des systèmes des équations :
\begin{align*}
  q_{1,t} & = \alpha_{1} + \beta P_{1,t} + \gamma_{1} Z_{1,t} + \epsilon_{1,t}  \\
  \vdots \\ 
  q_{N,t} & = \alpha_{N} + \beta P_{N,t} + \gamma_{N} Z_{N,t} + \epsilon_{1,t}  \\
  q_{1,t} & = a_1 + b P_{1,t} + c_1 X_{1,t} + u_{1,t} \\ 
  \vdots \\ 
  q_{N,t} & = a_N + b P_{N,t} + c_N X_{N,t} + u_{N,t} \\
\end{align*}
En simplifiant l'écriture nous pouvons la representer sous la forme suivante :
\begin{align*}
  q_{i,t} & = \alpha_{i} + \beta P_{i,t} + \gamma_{i} Z_{i,t} + \epsilon_{i,t} \\
  q_{i,t} & = a_i + b P_{i,t} + c_i X_{i,t} + u_{i,t}
\end{align*}
D'ici nous avons à notre disposition deux outils d'identification des effets étudiés :
\begin{itemize}
  \item Résolution du sytème par le passage aux équation structurelles.
  \item Doubles moindre carrés, lesquels on peut utiliser car on s'interesse principalement aux rôle de pésticides dans la production du vin ;
\end{itemize}

\subsubsection{Equations structurelles}
Ici nous suivons la même logique que lors d'éstilation du modèle 1.
Nous construisons deux équations structurelles sans des variables endogènes, cequi nous permettra d'identifier les effets d'intéret.
A partir du système :
\begin{align*}
  q_{i,t} & = \alpha_{i} + \beta P_{i,t} + \gamma_{i} Z_{i,t} + \epsilon_{i,t} \\
  q_{i,t} & = a_i + b P_{i,t} + c_i X_{i,t} + u_{i,t}
\end{align*}
Nous dérivons l'éstimateur pour la variable $P_{i,t}$ :
\begin{equation*}
  \alpha_{i} + \beta P_{i,t} + \gamma_{i} Z_{i,t} + \epsilon_{i,t} = 
  a_i + b P_{i,t} + c_i X_{i,t} + u_{i,t}
\end{equation*}
D'où :
\begin{equation*}
  P_{i,t} = \frac{\alpha_{i} - a_i}{b - \beta} + 
    \frac{\gamma_{i}}{b - \beta} Z_{i,t} + 
    \frac{(- c_i)}{b - \beta} X_{i,t} +
    \frac{\epsilon_{i,t} - u_{i,t}}{b - \beta}
\end{equation*}
Ce qu'on peut réécrire sous la forme simplifiée :
\begin{equation*}
  P_{i,t} = \pi_{1,i} + 
    \pi_{2,i} Z_{i,t} + 
    \pi_{3,i} X_{i,t} +
    v_{i,t}
\end{equation*}
D'ici nous pouvons dériver une équation structurelle pour la deuxième variable endogène $q_{i,t}$ :
\begin{equation*}
  q_{i,t} = [a_i + \frac{b (\alpha_{i} - a_i)}{b - \beta}] + 
    [\frac{b (\gamma_{i})}{b - \beta} Z_{i,t}] + 
    [c_i + \frac{b (- c_i)}{b - \beta} X_{i,t}] +
    [u_{i,t} + \frac{b (\epsilon_{i,t} - u_{i,t})}{b - \beta}]
\end{equation*}
Ou :
\begin{equation*}
  q_{i,t} = \theta{1,i} + 
    \theta{2,i} Z_{i,t}] + 
    \theta{3,i} X_{i,t}] +
    w_{i,t}
\end{equation*}
Ce qui nous permet de dériver les coefficients des équations de départ.

\subsubsection{Doubles moindres carrés}
Ici la logique est plus simple, car nous construisons un éstimateur IV pour une des variables éndogènes et on utilise cette éstimateur pour identifier les effets étudiés.
Nous pouvons construire un éstimateur pour la variable éndogene $P_{i,t}$ en utilisant les variables exogenes d'equaiton de la demande comme des instruments d'une façon suivante :
\begin{equation*}
  P_{i,t} = \psi_{1,i,t} Z_{i,t} + \psi_{2,i,t} X_{i,t} + e_{i,t}
\end{equation*}
Nous intégrons ensuite les résultats $\hat{P_{i,t}}$ dans l'équation de l'offre :
\begin{equation*}
  q_{i,t} = a_i + b \hat{P_{i,t}} + c_i X_{i,t} + u_{i,t}
\end{equation*}
Cet éstimateur donne des résultats moins fiable que celui d'avant, mais se prouve bequcoup plus simple à implementer.
Dans le cas, où $Z_{i,t}$ ne comprends que le revenu, le système est juste-identifié.

\subsection{Premiers résultats économétriques}
Dans cette séction nous allons presenter les prémiers résultats économétriques.
Lors de ces prémiéres éstimations nous supposons que les effets sont identiques pour tous les département (on fait la correction pour les effets fixes au niveau départemental quand même).

\subsubsection{Modèle 1 : les équations structurelles}

\subsubsection{Modèle 2 : les équations structurelles}
Ici nous presentons les résultats pour les éstimation des équations structurelles :
\FloatBarrier
\begin{table}[!htbp]
  \centering
  \begin{tabular}{@{\extracolsep{5pt}}lcc} 
    \\[-1.8ex]\hline 
    \hline \\[-1.8ex] 
     & \multicolumn{2}{c}{\textit{Dependent variable:}} \\ 
    \cline{2-3} 
    \\[-1.8ex] & qi & ipi \\ 
    \\[-1.8ex] & (1) & (2)\\ 
    \hline \\[-1.8ex]
     si & 0.348$^{***}$ & 0.298$^{***}$ \\ 
      & (0.052) & (0.074) \\ 
      & & \\ 
     iki & $-$0.127$^{**}$ & $-$0.179$^{**}$ \\ 
      & (0.064) & (0.091) \\ 
      & & \\ 
     ri & $-$3.338$^{*}$ & 2.336 \\ 
      & (1.933) & (2.736) \\ 
      & & \\ 
     Constant & 0.022 & 0.205$^{***}$ \\ 
      & (0.024) & (0.034) \\ 
      & & \\ 
    \hline \\[-1.8ex] 
    Observations & 345 & 345 \\ 
    R$^{2}$ & 0.154 & 0.061 \\ 
    Adjusted R$^{2}$ & 0.146 & 0.052 \\ 
    Residual Std. Error (df = 341) & 0.391 & 0.553 \\ 
    F Statistic (df = 3; 341) & 20.613$^{***}$ & 7.333$^{***}$ \\ 
    \hline 
    \hline \\[-1.8ex] 
    \textit{Note:}  & \multicolumn{2}{r}{$^{*}$p$<$0.1; $^{**}$p$<$0.05; $^{***}$p$<$0.01} \\ 
    \end{tabular} 
  \caption{Structural équation éstimation}
\end{table}
\FloatBarrier
La covariance entre les résidus de ces deux modèles est de : $0.7258$.
\par 
Pour le moment nous n'offrons pas la dérivation complete des effets des équations de départ.

\subsubsection{Modèle 2 : doubles moindres carrés}
Les estimation de l'équatoin de l'offre par doubles moindres carrées donne des résultats suivants :
\FloatBarrier
\begin{table}[!htbp]
  \centering
  \begin{tabular}{@{\extracolsep{5pt}}lc} 
    \\[-1.8ex]\hline 
    \hline \\[-1.8ex] 
     & \multicolumn{1}{c}{\textit{Dependent variable:}} \\ 
    \cline{2-2} 
    \\[-1.8ex] & qi \\ 
    \hline \\[-1.8ex] 
     ipi & $-$1.429 \\ 
      & (2.345) \\ 
      & \\ 
     si & 0.773 \\ 
      & (0.692) \\ 
      & \\
     iki & $-$0.383 \\ 
      & (0.434) \\ 
      & \\ 
     Constant & 0.315 \\ 
      & (0.481) \\ 
      & \\ 
    \hline \\[-1.8ex] 
    Observations & 345 \\ 
    R$^{2}$ & $-$5.793 \\ 
    Adjusted R$^{2}$ & $-$5.853 \\ 
    Residual Std. Error & 1.106 (df = 341) \\ 
    \hline 
    \hline \\[-1.8ex] 
    \textit{Note:}  & \multicolumn{1}{r}{$^{*}$p$<$0.1; $^{**}$p$<$0.05; $^{***}$p$<$0.01} \\ 
    \end{tabular} 
    \caption{IV estimation}
\end{table}
\FloatBarrier
Nous observons que les résultats obtenus ne sont pas significativement différents de 0, ce que peut s'expliquer par le fait, que l'offre de vin de table se comporte differement pour des différents départements.

\bibliography{biblio}

\end{document}