\PassOptionsToPackage{unicode=true}{hyperref} % options for packages loaded elsewhere
\PassOptionsToPackage{hyphens}{url}
%
\documentclass[11pt,ignorenonframetext,]{beamer}
\usepackage{pgfpages}
\setbeamertemplate{caption}[numbered]
\setbeamertemplate{caption label separator}{: }
\setbeamercolor{caption name}{fg=normal text.fg}
\beamertemplatenavigationsymbolsempty
% Prevent slide breaks in the middle of a paragraph:
\widowpenalties 1 10000
\raggedbottom
\setbeamertemplate{part page}{
\centering
\begin{beamercolorbox}[sep=16pt,center]{part title}
  \usebeamerfont{part title}\insertpart\par
\end{beamercolorbox}
}
\setbeamertemplate{section page}{
\centering
\begin{beamercolorbox}[sep=12pt,center]{part title}
  \usebeamerfont{section title}\insertsection\par
\end{beamercolorbox}
}
\setbeamertemplate{subsection page}{
\centering
\begin{beamercolorbox}[sep=8pt,center]{part title}
  \usebeamerfont{subsection title}\insertsubsection\par
\end{beamercolorbox}
}
\AtBeginPart{
  \frame{\partpage}
}
\AtBeginSection{
  \ifbibliography
  \else
    \frame{\sectionpage}
  \fi
}
\AtBeginSubsection{
  \frame{\subsectionpage}
}
\usepackage{lmodern}
\usepackage{amssymb,amsmath}
\usepackage{ifxetex,ifluatex}
\usepackage{fixltx2e} % provides \textsubscript
\ifnum 0\ifxetex 1\fi\ifluatex 1\fi=0 % if pdftex
  \usepackage[T1]{fontenc}
  \usepackage[utf8]{inputenc}
  \usepackage{textcomp} % provides euro and other symbols
\else % if luatex or xelatex
  \usepackage{unicode-math}
  \defaultfontfeatures{Ligatures=TeX,Scale=MatchLowercase}
\fi
\usetheme[]{Montpellier}
\usecolortheme{beaver}
% use upquote if available, for straight quotes in verbatim environments
\IfFileExists{upquote.sty}{\usepackage{upquote}}{}
% use microtype if available
\IfFileExists{microtype.sty}{%
\usepackage[]{microtype}
\UseMicrotypeSet[protrusion]{basicmath} % disable protrusion for tt fonts
}{}
\IfFileExists{parskip.sty}{%
\usepackage{parskip}
}{% else
\setlength{\parindent}{0pt}
\setlength{\parskip}{6pt plus 2pt minus 1pt}
}
\usepackage{hyperref}
\hypersetup{
            pdftitle={Etude des effets des pésticides dans la production des vins de table},
            pdfauthor={A. Blanc, N. Gusarov, S. Picon},
            pdfborder={0 0 0},
            breaklinks=true}
\urlstyle{same}  % don't use monospace font for urls
\newif\ifbibliography
\setlength{\emergencystretch}{3em}  % prevent overfull lines
\providecommand{\tightlist}{%
  \setlength{\itemsep}{0pt}\setlength{\parskip}{0pt}}
\setcounter{secnumdepth}{0}

% set default figure placement to htbp
\makeatletter
\def\fps@figure{htbp}
\makeatother

\usepackage{array}
\usepackage{multicol}

\title{Etude des effets des pésticides dans la production des vins de table}
\providecommand{\subtitle}[1]{}
\subtitle{Analyse empirique des marchés}
\author{A. Blanc, N. Gusarov, S. Picon}
\providecommand{\institute}[1]{}
\institute{Université Grenoble Alpes}
\date{11/12/2019}

\begin{document}
\frame{\titlepage}

\begin{frame}
\tableofcontents[hideallsubsections]
\end{frame}
\begin{frame}{Introduction}
\protect\hypertarget{introduction}{}

\end{frame}

\begin{frame}{Plan de la présentation}
\protect\hypertarget{plan-de-la-presentation}{}

\begin{itemize}
\tightlist
\item
  Présentation de la problématique
\item
  Présentation des données
\item
  Modélisation
\item
  Les résultats
\end{itemize}

\end{frame}

\begin{frame}{Le problème des pesticides}
\protect\hypertarget{le-probleme-des-pesticides}{}

\begin{itemize}
\tightlist
\item
  Présentation du problème des pésticides
\item
  Etat actuel
\item
  Comment combattre
\end{itemize}

\end{frame}

\begin{frame}{Le marché du vin français}
\protect\hypertarget{le-marche-du-vin-francais}{}

\begin{itemize}
\tightlist
\item
  Le marché commun
\item
  Utilisation des pésticides
\item
  Heterogénéité
\item
  Pourquoi vins de table
\end{itemize}

\end{frame}

\begin{frame}{Le Modèle théorique}
\protect\hypertarget{le-modele-theorique}{}

\begin{itemize}
\tightlist
\item
  Le rôle des pesticides dans la production du vin
\item
  Le rôle de la demande sur la production et l'offre en général
\item
  La formalisation et les équations
\end{itemize}

\end{frame}

\begin{frame}{Les données}
\protect\hypertarget{les-donnees}{}

\begin{itemize}
\tightlist
\item
  Dimentions :

  \begin{itemize}
  \tightlist
  \item
    Départements
  \item
    Années
  \end{itemize}
\item
  Les variables :

  \begin{itemize}
  \tightlist
  \item
    Pésticides (quantités)
  \item
    Vins (quantités produits, prix)
  \item
    Variables de controle (revenus, surface cultivé)
  \end{itemize}
\end{itemize}

\end{frame}

\begin{frame}{Les statistiques déscriptives}
\protect\hypertarget{les-statistiques-descriptives}{}

\begin{itemize}
\tightlist
\item
  Between and within variance par variable
\item
  Bivariate plots with support regressions (lokk for ggplot examples)
\item
  Covariance analysis
\end{itemize}

\end{frame}

\begin{frame}{Modèlisation}
\protect\hypertarget{modelisation}{}

\begin{itemize}
\tightlist
\item
  Explication de la méthode utilisée

  \begin{itemize}
  \tightlist
  \item
    Panel data
  \item
    AIDS model
  \end{itemize}
\item
  Limites du modèle
\end{itemize}

\end{frame}

\begin{frame}{Résultats d'estimation}
\protect\hypertarget{resultats-destimation}{}

\begin{itemize}
\tightlist
\item
  Les coefficients estimés avec la variance
\item
  Etude des erreurs
\item
  Vérification des hypothèses (5 hypothèses) :

  \begin{itemize}
  \tightlist
  \item
    La moyenne nulle des herreurs
  \item
    Homoscedacité
  \item
    Autocorrélation
  \item
    Spécification du modèle
  \item
    \ldots{} (à voir)
  \end{itemize}
\end{itemize}

\end{frame}

\begin{frame}{Conclusions}
\protect\hypertarget{conclusions}{}

\begin{itemize}
\tightlist
\item
  Le rôle des pésticides
\item
  Le marché du vin
\item
  Validité
\item
  Limitations
\item
  Ouverture
\end{itemize}

\end{frame}

\begin{frame}{Bibliographie}
\protect\hypertarget{bibliographie}{}

\begin{itemize}
\tightlist
\item
  Inclure seulement les articles importants
\item
  Faire des réferences et mentionner ces articles dans la partie
  théorique
\end{itemize}

\end{frame}

\end{document}
