\documentclass[11pt,]{article}
\usepackage{lmodern}
\usepackage{amssymb,amsmath}
\usepackage{ifxetex,ifluatex}
\usepackage{fixltx2e} % provides \textsubscript
\ifnum 0\ifxetex 1\fi\ifluatex 1\fi=0 % if pdftex
  \usepackage[T1]{fontenc}
  \usepackage[utf8]{inputenc}
\else % if luatex or xelatex
  \ifxetex
    \usepackage{mathspec}
  \else
    \usepackage{fontspec}
  \fi
  \defaultfontfeatures{Ligatures=TeX,Scale=MatchLowercase}
\fi
% use upquote if available, for straight quotes in verbatim environments
\IfFileExists{upquote.sty}{\usepackage{upquote}}{}
% use microtype if available
\IfFileExists{microtype.sty}{%
\usepackage{microtype}
\UseMicrotypeSet[protrusion]{basicmath} % disable protrusion for tt fonts
}{}
\usepackage[margin=1in]{geometry}
\usepackage{hyperref}
\hypersetup{unicode=true,
            pdftitle={Etude des effets des pésticides dans la production des vins de table},
            pdfauthor={Arnaud Blanc, Nikita Gusarov, Sasha Picon},
            pdfborder={0 0 0},
            breaklinks=true}
\urlstyle{same}  % don't use monospace font for urls
\usepackage{longtable,booktabs}
\usepackage{graphicx,grffile}
\makeatletter
\def\maxwidth{\ifdim\Gin@nat@width>\linewidth\linewidth\else\Gin@nat@width\fi}
\def\maxheight{\ifdim\Gin@nat@height>\textheight\textheight\else\Gin@nat@height\fi}
\makeatother
% Scale images if necessary, so that they will not overflow the page
% margins by default, and it is still possible to overwrite the defaults
% using explicit options in \includegraphics[width, height, ...]{}
\setkeys{Gin}{width=\maxwidth,height=\maxheight,keepaspectratio}
\IfFileExists{parskip.sty}{%
\usepackage{parskip}
}{% else
\setlength{\parindent}{0pt}
\setlength{\parskip}{6pt plus 2pt minus 1pt}
}
\setlength{\emergencystretch}{3em}  % prevent overfull lines
\providecommand{\tightlist}{%
  \setlength{\itemsep}{0pt}\setlength{\parskip}{0pt}}
\setcounter{secnumdepth}{0}
% Redefines (sub)paragraphs to behave more like sections
\ifx\paragraph\undefined\else
\let\oldparagraph\paragraph
\renewcommand{\paragraph}[1]{\oldparagraph{#1}\mbox{}}
\fi
\ifx\subparagraph\undefined\else
\let\oldsubparagraph\subparagraph
\renewcommand{\subparagraph}[1]{\oldsubparagraph{#1}\mbox{}}
\fi

%%% Use protect on footnotes to avoid problems with footnotes in titles
\let\rmarkdownfootnote\footnote%
\def\footnote{\protect\rmarkdownfootnote}

%%% Change title format to be more compact
\usepackage{titling}

% Create subtitle command for use in maketitle
\providecommand{\subtitle}[1]{
  \posttitle{
    \begin{center}\large#1\end{center}
    }
}

\setlength{\droptitle}{-2em}

  \title{Etude des effets des pésticides dans la production des vins de table}
    \pretitle{\vspace{\droptitle}\centering\huge}
  \posttitle{\par}
  \subtitle{Analyse empirique des marchés}
  \author{Arnaud Blanc, Nikita Gusarov, Sasha Picon}
    \preauthor{\centering\large\emph}
  \postauthor{\par}
      \predate{\centering\large\emph}
  \postdate{\par}
    \date{25/12/2019}

\usepackage{setspace}

% to make the first rows bold in tables
\usepackage{longtable}
\usepackage{tabu}
\usepackage{booktabs}

% Floats
\usepackage{morefloats}
\usepackage{float}
\usepackage{placeins}

% highlighting
\usepackage{soul}

% Short toc
\usepackage{shorttoc}
\setcounter{tocdepth}{1}

% referencing mutliple things with a single command - \cref
\usepackage{cleveref}

% Change section names style
\usepackage[dvipsnames]{xcolor}
% \usepackage{sectsty}

% \sectionfont{\color{Green}}  % sets colour of sections
% \subsectionfont{\color{Green}}  % sets colour of sub
% \subsubsectionfont{\color{Green}}  % sets colour of subsub

% this makes dots in table of contents
% \renewcommand{\cftsecleader}{\cftdotfill{\cftdotsep}}
% to change the title of contents
% \renewcommand{\contentsname}{Whatever}

% line numbers for review purposes
% this package might not be available in default latex installation 
% get it by 'sudo tlmgr install lineno'
%\usepackage{lineno}
%\linenumbers

% Array
\usepackage{array}

% Multiple columns
\usepackage{multicol}

% Image insertion and colors
\usepackage{graphicx}

% to be able to include latex comments
\newenvironment{dummy}{}{}

% maketitle definition
\makeatletter
\def\@maketitle{
    \pagenumbering{gobble}
    \raggedright
    \includegraphics[height = 40mm]{univlogo.jpg} 
    \begin{center}
        \vspace*{\fill}
            {\Huge \@title}\\
            \par
            \rule{5cm}{0.4pt}
            \par
            %\textbf{Rapport de stage}\\[10mm]
            {\Large \@author}\\[10mm]
        \vspace*{\fill}
    \end{center}
    {\large Tuteur : }\\
    \hspace{10mm} {\large Adélaïde Fadhuile}\\
    \vspace{10mm}
    {\large Niveau d'études : }\\
    \hspace{10mm} {\large Master 2}\\
    {\large Parcours : }\\
    \hspace{10mm} {\large Chargé d'études économiques et statistique}\\
    \vspace{20mm}
    \begin{center}
        {\large Université Grenoble Alpes}\\
        {\large Faculté d'économie et gestion}\\
        \vspace{5mm}
        2019 - 2020\\
    \end{center}
    \clearpage
}
\makeatother

\begin{document}
\maketitle


\hypersetup{linkcolor = black}
\pagenumbering{roman}

\tableofcontents

% \newpage

% % list of figures have to be added manually to table of contents
% \listoffigures 

% \newpage
% \listoftables

% \doublespacing

\newpage

\pagenumbering{arabic}
\hypersetup{linkcolor = blue}

\hypertarget{introduction}{%
\section{Introduction}\label{introduction}}

Aujourd'hui, l'utilisation des pesticides est un problème majeur de
l'agriculture.\\
Celle-ci utilise la majeure partie des pesticides en France. Il s'agit
d'un enjeu majeur du développement durable car ils ont un impact
important sur les risques environnementaux et sanitaires.

Les pesticides sont utilisés dans l'agriculture pour protéger la
production. Il est supposer que les pesticides servent à protéger les
rendements. En effet, les aléas climatiques influencent sur le
développement de champignons ou de maladie. Ainsi, les pesticides
permettent de protéger les cultures contre les aléas climatiques et de
ne pas perdre de production.

Dans ce travail nous cherchons à comprendre et estimer les effets des
pésticides sur le marché des vins simples. De cette façon nous
chercherons à étudier l'équilibre sur le marché des vins simples ce qui
est sensé de nous donner des résultats plus précis et fiables.

\hypertarget{les-pesticides}{%
\section{1. Les pesticides}\label{les-pesticides}}

\noindent

\rule[0.5ex]{\linewidth}{1pt}

\textcolor{red}{Mettre des sources partout !}

\noindent

\rule[0.5ex]{\linewidth}{1pt}

Pour lutter contre l'utilisation des pesticides l'Etat Français et
l'union européenne ont mis en place des mesures. Ainsi, l'Etat Français
lors du grenelle de l'environnement de 2006 a fixé ces objectifs. Ainsi,
le plan ECOPHYTO 2018 visait à réduire de 50\% l'utilisation des
pesticides de synthèse. Le deuxième objectif est le passage en
agriculture biologique à 6\% de la surface agricole utilisée en 2010 et
vise 20\% en 2020.

En 2008, les 30 produits les plus toxiques les plus toxiques sont
interdits. Une taxe sur les phytosanitaires a aussi été mise en place.
Cette taxe est croissante avec leur niveau de toxicité. Cette taxe
devait augmenter au fil des années et l'octroi de crédits d'impôt en
faveur de l'agriculture biologique.

Malgré tous ces efforts, l'utilisation des pesticides perdurent. La
France enregistre une hausse des ventes de produits phytosanitaires de
10000 tonnes, entre 2011 et 2016. En 2008, le nombre de doses unités a
été créé pour enregistrer l'évolution de la demande de pesticide. On
remarque que les doses utilisées augmentent de 12\% en 2014-2016 par
rapport à 2009-2011.

\hypertarget{etat-actuel}{%
\subsection{Etat actuel}\label{etat-actuel}}

Contrairement aux attentes des autorités, on ne remarque aucune baisse
de l'utilisation de pesticides. Le Nodu a connu une hausse de 23\% entre
2008 et 2017. Certaines critiques ont été faites sur l'utilisation du
Nodu. Il est possible d'utiliser le nombre de substances actives
utilisées. Mais, cet indicateur connaît lui aussi une hausse de 15\%
entre 2011 et 2017.

Néanmoins, les politiques ont quand même eu quelques effets positifs,
puisque l'achat des produits les plus dangereux baisse de 6\% en 2017.\\
Les grandes cultures sont les premières utilisatrices de pesticides.
Elles représentent 67,4\% de l'utilisation de pesticides. La deuxième
culture est celle de la vigne ce qui représente 14,4\% des pesticides
utilisés.

\hypertarget{comment-baisser-lutilisation-de-pesticides}{%
\subsection{Comment baisser l'utilisation de
pesticides}\label{comment-baisser-lutilisation-de-pesticides}}

Afin de baisser l'utilisation des pesticides, des méthodes de cultures
ont été développées pour baisser l'utilisation des pesticides. Il est
possible d'utiliser différents mode de culture. On peut en retenir trois
principaux.

Le premier est l'agriculture intensive. Elle ne limite pas le recours
aux pesticides.

Le deuxième est l'agriculture raisonnée. Elle limite le recours aux
pesticides en fonction de seuils.

Le troisième niveau est l'agriculture biologique. Elle supprime les
traitements avec des produits phytosanitaires de synthèse.

Les professionnels proposent de commencer par utiliser l'agriculture
raisonnée qui permet de réduire les doses de pesticides légales. Ensuite
l'agriculture doit se déplacer vers l'agriculture biologique qui
n'utilise aucun produit phytosanitaire de synthèse.

\hypertarget{le-marche-du-vin-francais}{%
\section{2. Le marché du vin français}\label{le-marche-du-vin-francais}}

La France est l'un des principaux producteurs et vendeurs de vin dans le
monde. En effet, la France représente 10\% de la surface de vigne dans
le monde. La surface de vigne française se répartit dans 65 des 95
départements de la métropole. En France, il y a plus de 750000 hectares
de vignes qui sont exploitées en 2018.

Ainsi, en France, une exploitation agricole sur cinq est une
exploitation viticole. Cela représente 85000 exploitations. La
production de vins en France, représentait 4,6 milliard de litres. Cela
représentait plus de 17\% de la production totale de vin. En volume de
production la France se place donc en deuxième position derrière le
volume de production de l'Italie. 3\% de la surface agricole est
consacrée à la production de vin. Néanmoins, le vin représente 15\% de
la production agricole en valeur.

Du côté du consommateur, la France est le deuxième pays consommateur de
vin derrière les Etats Unis. En effet, la consommation de vin en France
représentait plus de 3,5 milliards de bouteille, en 2018. Néanmoins, on
remarque une baisse de la consommation Française depuis une trentaine
d'année.

\hypertarget{le-probleme-dheterogeneite}{%
\subsection{Le problème
d'heterogénéité}\label{le-probleme-dheterogeneite}}

Il existe une forte hétérogénéité entre les différents labels mais aussi
à l'intérieur de ces labels.

Dans le commerce du vin, il est courant de diviser les vins en deux
grandes classes en fonction de leurs prix (Cembalo, Caracciolo, and
Pomarici 2014) :

\begin{itemize}
\tightlist
\item
  les vins de qualité inférieure, les moins chers avec les
  caractéristiques de qualité de base ;
\item
  les vins de qualité supérieure plus chers, dotés de caractéristiques
  qualitatives complexes et d'une image de grande valeur.
\end{itemize}

De plus, pour les vins français, selon Steiner (2004), le système
européen de classification des ``\emph{vins de qualité produits dans
certaines régions}'' (VQPRD) contient à la fois des vins AOC et des
``\emph{vins de haute qualité provenant d'un vignoble régional agréé}''
(VDQS). Les vins de cépage appartiennent à la catégorie des vins autres
que VQPRD, qui comprend les \textbf{vins de table} et les
\textbf{vins de pays}.

En tenant compte des spécificités du marhcé du vin français, nous
utilisons la méthodologie du ministère d'agriculture et divisons le
marché en deux parties :

\begin{itemize}
\tightlist
\item
  La gamme haute (les vins IGP, vendus dans des magasins spécifiques) ;
\item
  La gamme basse (les vins non IGP, vendus en grands surfaces).
\end{itemize}

La première partie est soumise à des règlements spécifiques :
limitations des quantités produites, origine contrôlé, un caractère de
la demande spécifique. La deuxième, c'est-à-dire le marché des vins
moins chers, est aussi complexe. Les produits classés dans cette
catégorie sont susceptibles d'avoir un certain degré d'hétérogénéité,
comme cela a été montré par Cembalo, Caracciolo, and Pomarici (2014).

\hypertarget{les-vins-de-table}{%
\subsection{Les vins de table}\label{les-vins-de-table}}

Ces vins sans indication géographique (sans IG) ont vu leurs
transactions augmenter en volume pour toutes les couleurs. Ainsi, on
remarque que pour les vins rouges les transactions ont augmenté de 10\%,
pour les rosées la hausse représentait 52\%, pour les vins blancs les
volumes de transactions ont presque été doublé. Néanmoins on remarque
egalement une baisse des cours des vins sans indication géographique.

En effet, on remarque que les prix moyens pour les vins rouges et rosées
sans indication géographique baisse de 3\%. Le prix moyen des vins
blancs baisse quand à eux de 12\%, pour la campagne 2019/2020. Sur les
deux mois de campagne, les échanges de Vin sans indication géographique
est de 142 milliers d'hectolitres. Cela correspond à une hausse de 39\%
par rapport à la campagne précédente. Les ventes représentent 92
milliers d'hectolitres.

La tendance sur le marché des vins sans indication géographique
s'explique par une forte hausse des vins blancs. En effet, ceux-ci
connaissent une hausse de près de 28 milliers d'hectolitres, soit une
hausse de 232\% vis-à-vis de la campagne de 2018-2019. Les vins rosés
connaissent également une hausse. Néanmoins, celle-ci reste modeste
puisque les ventes augmentaient de 61\% par rapport à la campagne
2018/2019. En même temps, les ventes de vins rouges ont légèrement
baissé. Le cours des Vins sans indication géographique baisse par
rapport à la campagne précédente.

Lors de la campagne 2018/2019, les ventes de vins en grande distribution
sont en baisse. Cela peut s'expliquer par une hausse des prix moyens.
Les ventes de vins représentent 8,7 millions d'hectolitres et un chiffre
d'affaires de 4,1 milliards d'euros avec un prix moyen de 4,73
euro/litre. La baisse de la consommation de vins rouges s'aggrave avec
une baisse de 8\% par rapport à la campagne de 2017/2018. Les vins
blancs connaissent aussi une faible baisse de 1,2\% en volume par
rapport à la consommation de la campagne précédente. Pour finir, les
ventes de vins rosés ont baissé lors de la campagne 2018/2019. En effet,
on enregistre une baisse de 3,9\% en volume par rapport à la campagne
2017/2018. La consommation de vin sans indication géographique est de
6\% en volume contre 3\% en valeur. Les ventes de vins sans indications
géographiques sont en légère hausse dans la campagne 2018/2019 par
rapport à la campagne 2017/2018.

Dans notre étude, nous traitons uniquement les vins simples (non IGP).
La situation sur ce marché est sensée influencer l'utilisation des
pesticides, car les volumes de productions sont plus significatives que
pour le marché des vins IGP.

Suivant le raisonnement des chercheurs (Cembalo, Caracciolo, and
Pomarici 2014), dans une catégorie de vin avec une fourchette de prix
étroite, il existe une homogénéité presque parfaite due à des vins ayant
des attributs intrinsèques simples, une complexité de qualité médiocre
et donc une différenciation peu marquée.

Cela nous permet d'analyser le marché par département est non par des
marques/produits.

\noindent

\rule[0.5ex]{\linewidth}{1pt}

\textcolor{red}{Ajouter des articles proches par la méthodologie à notre }

\begin{itemize}
\tightlist
\item
  \textcolor{red}{le cas du modèle simple,  }
\item
  \textcolor{red}{le cas du marchéliée, }
\item
  \textcolor{red}{le cas des clusters.}
\end{itemize}

\noindent

\rule[0.5ex]{\linewidth}{1pt}

\hypertarget{utilisation-des-pesticides-dans-la-viticulture}{%
\subsection{Utilisation des pesticides dans la
viticulture}\label{utilisation-des-pesticides-dans-la-viticulture}}

Les phytosanitaires sont très utilisés dans les cultures comme la
viticulture. Il s'agit donc d'un intrant important pour la production de
vin. Ainsi, la viticulture utilisait 15\% de produit phytosanitaire. La
pression sanitaire varie selon les productions et elle est
particulierement forte en viticulture. De la même façon, la pression
phytosanitaire varie selon les régions. Ainsi, pour la vigne l'IFT varie
de 7 en Provence à 22 en Champagne.

\hypertarget{le-cadre-theorique}{%
\section{3. Le cadre théorique}\label{le-cadre-theorique}}

\hypertarget{les-hypotheses-theoriques}{%
\subsection{Les hypothèses théoriques}\label{les-hypotheses-theoriques}}

\noindent

\rule[0.5ex]{\linewidth}{1pt}

\textcolor{red}{Ajouter les réferences ...}

\noindent

\rule[0.5ex]{\linewidth}{1pt}

Comme proposé dans la littérature, notre étude sur les vins non coûteux
(non IGP) est effectué au niveau du pays Cembalo, Caracciolo, and
Pomarici (2014) pour deux raisons. D'abord, les prix de vente moyens des
marchés sont diffèrent en raison des droits de douane à l'importation et
des taxes à la consommation différents (Anderson, Nelgen, and others
2011). De plus, la perception des produits de consommation varie d'un
pays à l'autre (MÄKELÄ et al. 2006).

Rachat du vin par les enseignes (grand surfaces) \ldots{} KREMER and
VIOT (2004)

La plupart des bouteilles achetées sont achetées dans la grande
distribution. Néanmoins, dans un souci de simplicité nous estimerons que
les consommateurs achètent leurs bouteilles directement auprès du
viticulteur. Donc nous supprimerons tous les intermédiaires entre le
producteur et le marché final.

Quand aux exportations et les importations, n'ayant pas la possibilité
contrôler le montant des vins non IGP exportés/importés, nous laissons
ces effets au terme d'erreur. Nous ignorons les interactions
internationales completement.

Facteurs de production \ldots{} Laporte and PICHERY (1996)

Les coûts des viticulteurs \ldots{} Laporte and PICHERY (1996)

Facteurs influençant le prix \ldots{} Outreville (2010)

Avant de conclure, nous proposons au lecteur une liste exhaustive des
suppositions sur le comportement du marché des vins simples.
Premierement, nous supposons que chaque département à une fonction de
production unique détérminé par des spécificités historiques, les
traditions, la législation, le térroir, ainsi que des conditions
météorologiques et géographiques. Les effets sont fixes au niveau
départamental et peuvent être isolés par des transformations spécifiques
des données (ex : une transformation Within). Deuxièmement, la quantité
vendu sur le marché départamental est consommé au sein du même
département. C'est une hipothèse trop réstrictive, qui nous eloigne de
la réalité, mais nous devrions l'adopté si nous voulons integrer les
rélations entre l'offre et la demande dans notre modèle. Afin de
vérifier cette hypothèse nous allons construire deux modèles differents.
Finalement, les effets qu'on vise à estimer sont des effets moyens au
niveau départamental. C'est à dire nous allons obtenir un estimateur des
effets moyens pour l'ensemble des département inclus dans notre analyse,
ou des effets moyens au sein des groupes des département, si nous
revelons des differences significatives entre les départements. Un autre
modèle nous permettra de vérifier et justifier cette hypothèse.

En ce qui concerne les pesticides, nous supposons d'abord, que
l'utilisation des pésticides par les viticulteurs est entrélié avec la
demande sur le vin et les préferences des consomamteurs. De plus, nous
posons, que la demande des pesticides est inélastique au prix, ce qui
nous permet d'exclure les intéractions entre les fournisseurs des
pésticides et les agriculteurs de notre analyse. C'est-à-dire, la
quantité de pesticides utilisés par les agriculteurs correspond
seulement à leurs intentions et besoins.

Pour sommariser cette partie, on résume que ce travail va porter sur les
effets des pésticides sur l'offre des vins simples. Nous allons tester
certains hypothèses sur le comportement et l'organisation des rélations
sur le marché des vins simples en comparant des differents modèles.
Puis, nous pourrions choisir entre ces modèles differents un le plus
vraisamblable, qui nous servira pour répondre à la question de
récherche.

\hypertarget{formalisation}{%
\subsection{Formalisation}\label{formalisation}}

En formalisant notre modèle théorique de base, nous posons, que l'offre
agregée pour toute la France est donnée identiquement par l'équation
suivante :

\begin{equation}
    Qo = \sum_{i = 1}^{N} qo_i
\end{equation}

Avec la quantité offerte déterminé par des contraintes de production et
le prix sur le marché :

\begin{equation}
    qo_i = a_i + b_i Po_i + c_i X_i
\end{equation}

Où \(X\) est un vecteur des variables explicatives influençant la
production. Dans le cas le plus simple nous ne prenons en compte que les
quantités des pesticides utilisées et la surface disponible, alors
l'effet \(c_{i1} : c_i = (c_{i1}, c{i2})\) represente l'effet
d'utilisation des pésticides dans la production du vin sur l'offre du
dérnier.

Cette équation déjà en soit permet d'estimer les effets d'utilisation
des pésticides sur le marché du vin. Appelons cette modèle théorique M1
pour le réferencier en futur, nous permettant de distinguer le cas sans
intéractions simultanées entre l'offre et la demande.

Il faut tenir compte que de cette façon nous ignorons plusieurs effets
pervers, tels que :

\begin{itemize}
\tightlist
\item
  La structure du marché interne de la France ;
\item
  La mobilité des produits finis entre des differents départements ;
\item
  L'exportation et l'importation du vin.
\end{itemize}

Toutefois, ces résultats ne séront valables que dans la situation où la
quantité du vin simple offerte sur le marché est déterminée seulement
par le producteur et n'est pas entrélié avec la demande. Comme nous
avons vu dans la séction précedente, la demande peut influencer les
décisions des viticulteurs (ex: le choix de la procedure téchnique à
suivre, d'utiliser ou non les pésticides, etc). Dans le cas pareil, nous
dévrions prendre en compte les intéractions entre l'offre et la demande.
A ce but nous introduisons également la demande dans notre analyse.

La demande agregée du vin en France peut s'écrire sous la forme suivante
:

\begin{equation*}
    Qd = \sum_{i = 1}^{N} qd_i 
\end{equation*}

Où \(i \in \{1, ..., N\}\) sont des départements, chacun ayant sa propre
fonction de la demande unique :

\begin{equation*}
    qd_i = \alpha_i + \beta_i Pd_i + \gamma_i Z_i 
\end{equation*}

Avec \(Z\) étant l'ensemble des variables ayant une influence sur la
demande du vin, dans le cas le plus simple nous n'utilisons que les
revenus (c'est une des variables les plus utilisées dans des études
empiriques sur le marché du vin).

Pour intégrer cette information dans notre \emph{framework} analytique,
nous dévons construire une système d'équations. Il y existe plusieures
façons de le faire.

Dans le prémièr cas, nous pouvons essayer de capter les effets au niveau
national. Pour le faire nous réecrivons les deux équation (de la demande
et de l'offre respectivement) sous la forme suivante :

\begin{equation*}
    Q_o = \sum_{i = 1}^{N} (a_i + b_i Po_i + c_i X) = \sum_{i = 1}^{N} a_i + \sum_{i = 1}^{N} b_i Po_i + \sum_{i = 1}^{N} c_i X
\end{equation*}

\begin{equation*}
    Qd = \sum_{i = 1}^{N} ( \alpha_i + \beta_i Pd_i + \gamma_i Z_i ) = \sum_{i = 1}^{N} \alpha_i + \sum_{i = 1}^{N} \beta_i Pd_i + \sum_{i = 1}^{N} \gamma_i Z_i
\end{equation*}

Ce qui nous produira un système des deux équations, avec \(Qd = Qo\)
dans la situation d'équilibre :

\begin{align*}
    Qd & = \sum_{i = 1}^{N} \alpha_i + \sum_{i = 1}^{N} \beta_i Pd_i + \sum_{i = 1}^{N} \gamma_i Z_i \\
    Qo & = \sum_{i = 1}^{N} a_i + \sum_{i = 1}^{N} b_i Po_i + \sum_{i = 1}^{N} c_i X
\end{align*}

Neanmoins, ce cas se réleve d'être très complex. D'abord, les effets
peuvent être differents pour tous les départements, ce qui nous conduira
à une augmentation dans le nombre des paramètres à estimer
significative. De plus, même si tous les effets sont identiques pour
l'ensemble des départements, des contraintes au niveau des données
peuvent se reveler d'être trop restrictives en réduisant au néant la
puissance statistique de notre éstimateur (ex : le nombre des
observation par années très faible). Dans le deux cas nous faisons face
à un impace.

Une des modification possibles dans ce cas sera l'introduction d'une
contrainte supplementaire au niveau de la demande sur le vin de table.
Afin de pouvoir identifier les effets de toutes les variables par un
système d'équations, nous pouvons supposer, que tout le vin produit dans
un département est consommé dans le même department. Dans ce cas nous
pourrions obtenir des estimateurrs pour les effets moyens au niveau
départemental. Toutefois, c'est une supposition forte, laquelle nous
éloigne de la réalité.

Théoriquement, nous pouvons tout de méme ignorer ces effets, car nous
visons à estimer les effets moyens pour tous les départements. De cette
façon, lors d'aggregation des effets au niveau national en éstimant le
coefficient moyen unique pour tous les départements nous allons mitiger
les biais possibles.

Alors,nous pouvons réécrire notre système d'equations sous la forme
suivante :

\begin{align*}
  qd_i & = \alpha_{i} + \beta Pd_{i,d} + \gamma Z_{i} \\
  qo_i & = a_i + b Po_{i,o} + c X_{i} \\ 
\end{align*}

Où \(qd_i = qo_i\) et \(Pd_i = Po_i\), ce qui permet de rélier les
équations au niveau départemental. Les coefficients \(b\), \(c\),
\(\beta\) et \(\gamma\) sont supposé fixes pour tous les départements
nous donnent un estimateur des effets moyens au niveau de la France.
L'effet des pésticides dans la production du vin serons captés par le
terme \(c_{1} : c = (c_{1}, c{2})\) dans ce cas.

Néanmoins, nous nous posons la question, comment réagir dans le cas où
les effets sont differents pour des differents département suite à des
spécificité des marché locaux, géographiques ou autres ? On peut
supposer, qu'il existent au moins quelques groupes majeures ayant des
caractéristiques et comportements similaires. Dans ce cas nous pourrions
construire des clusters, qui regrouppont des départements ayant des
caractéristiques idéntiques. Cela nous permettra de modèliser les effets
moyens par cluster en réduisant les biais eventuels.

Ce système peut être formalisé par \(K\) systèmes d'équations suivantes
:

\begin{align*}
  qd_{i_{c = const}} & = \alpha_{i_{c = const}} + \beta_{c = const} Pd_{i_{c = const},d} + \gamma_{c = const} Z_{i_{c = const}} \\
  qo_{i_{c = const}} & = a_{i_{c = const}} + b_{c = const} Po_{i_{c = const},o} + c_{c = const} X_{i_{c = const}} \\ 
\end{align*}

Où \(c\) décrive l'appartenance des départements à un des groupes
(clusters).

\hypertarget{les-donnees}{%
\section{4. Les données}\label{les-donnees}}

Avant de passer à la discussion des modèles économétriques il nous faut
prendre connaissance de la nature des données en notre disposition. Dans
cette partie de notre travail nous allons presenter la base des données
utilisé lors de cette étude. Nous commencerons par une presentation des
sources et des types des données extraits de ces sources. Puis, nous
procederons avec la déscription des méthodes et thécniques utilisées
pour transformer ces données et les rendre traitables. Finalement, nous
presenterons un dictionnaire des variables pour nos bases des données.

\hypertarget{sources-des-donnees}{%
\subsection{Sources des données :}\label{sources-des-donnees}}

Nous avons utilisé les bases des données suivantes pour notre analyse :

\begin{itemize}
\tightlist
\item
  Les données de ventes de pesticides par département (INERIS)
\item
  Les données sur les prix du vin (France Agrimer)
\item
  Les données sur la population (INSEE)
\item
  Les données sur la production de vin (SSM Finances Publiques)
\end{itemize}

\hypertarget{les-variables-utilisees-pour-notre-modele}{%
\subsection{Les variables utilisées pour notre
modèle}\label{les-variables-utilisees-pour-notre-modele}}

\noindent

\rule[0.5ex]{\linewidth}{1pt}

\textcolor{red}{Réverifier tous les sources et la naure des données ...}

\textcolor{red}{Expliciter la procedure de création des variables}

\textcolor{red}{Preciser les effets attendus des variables}

\textcolor{red}{Discuter les externalités (ou c'est mieux de l'inclure dans la partie théorique ? ou contextualisation ? A VOIR)}

\noindent

\rule[0.5ex]{\linewidth}{1pt}

Dans notre étude nous faisons face à un problème avec deux variables
endogènes et trois variables exogènes.

Variables endogènes : - la quantité totale produite de vin rouge et
blanc non IG par département (en hectolitres, en log), - le prix moyen
des vins rouges-blancs (idice, en log).

Variables exogènes : - le revenu médian par département (en euros par
personne par année, en log), - la surface agricole destinée aux vins de
table (en hectares, en log), - la quantité des pesticides utilisés sur
la vigne (indice, en log).

Au niveau des pesticides, on va s'intéresser plus particulièrement aux
quantités de produits vendus par département entre 2009 et 2017 utilisés
principalement sur les cultures viticoles. Il faut faire preuve de
vigilance sur le conditionnement des produits qui n'est pas exprimé dans
la même unité au sein de cette base : en litres ou en kilos. Dans notre
étude nous allons étudier l'impact de la masse totale des pésticides
utilisés. Pour pouvoir le faire, nous créons un indice qui permet de
prendre en compte les évolutions des differents types des produits à la
fois. Nous créons un indice simple :

\begin{equation*}
  P = \frac{\sum_j p_{j, t} q_{j, t}}{\sum_j p_{j, 0} q_{j, 0}}
\end{equation*}

Avec \(j\) désignant le produit \(j\), et \(p\) étant un coefficient de
pondération (dans le cas le plus simple \(p = 1\)).

En ce qui concerne les données sur le prix du vin, on s'intéresse
principalement au prix moyen des vins rouge- rosés et blancs sans IG
(Indication Géographique) sur la période 2009-2017. Ces prix sont
déflatés par l'indice des prix à la consommation (base 100 en 2014). On
ne considère ici que le prix moyen déflaté au niveau national. Dans le
deuxième modèle nous avons besoin de créer artificiellement un
estimateur qui va varier par département. Dans ce but nous créons
l'indice de prix du vin de table départementale, calculé de façon
suivante :

\begin{equation*}
  P = \frac{p_{rouge, t} q_{rouge, t} + p_{blanc, t} q_{blanc, t}}{p_{rouge, 0} q_{rouge, 0} + p_{blanc, 0} q_{blanc, 0}}
\end{equation*}

Avec \(t\) étant l'anée au période \(t\).

Au niveau des données sur la population, la variable qui nous intéresse
ici est relative au niveau de revenu, exprimée au niveau départemental
(laquelle, si besoin nous pourrions facilement aggréger au niceau
national). Plus précisément, on va utiliser le revenu médian par
département. Il est aussi déflatée de l'indice des prix à la
consommation (base 100 en 2014).

Toutes les variables subissent une transformation logarithmique, ce qui
nous permet d'interpreter les effets estimés plus facilement. Pour un
modèle logarithmique nous pourrions traiter les estimateurs obtenus
comme l'elasticité de la demande/l'offre par rapport à des facteurs
differents. Ainsi, nous cherchons particulierement l'elasticité de
quantité offerte sur le marché par rapport à la quantité des pésticides
utilisés.

Les propriétés de ces données sont suivantes :

\begin{itemize}
\tightlist
\item
  Toutes les variables varient par département et par année.
\item
  Le période temporelle comprise dans notre échantillon est de 2012 à
  2016.
\item
  Nous ne considérons que les régions produisant du vin.
\item
  Nous éliminons les effets fixes pour en substrayant les moyennes
  départamentales.
\item
  Données en panel ``cylindrées''.
\item
  Nombre des individus large (69 départements, qui produisent le vin
  simple et qui utilisent des pésticides) et le nombre des périodes
  pauvre (5 périodes).
\end{itemize}

\hypertarget{letude-statistique}{%
\section{5. L'étude statistique}\label{letude-statistique}}

Dans cette partie de travail nous allons explorer les données
collectées.

De l'étude de la variance pour les données en panel avec des
statistiques générales, nous passerons vers l'étude des interdependances
des variables. Puis, nous allons finir avec étude des donnée alternées
par une transformation \textbf{within}.

\hypertarget{visualisation-au-niveau-de-la-france}{%
\subsection{Visualisation au niveau de la
France}\label{visualisation-au-niveau-de-la-france}}

Pour le prémier analyse il peut être interessant de voir la situation du
point du vue géographique. Nous visualisons les valuers moyens par
département des differentes variables (une partie des répresentation se
trouve dans l'annexe X).

D'abord nous étudions le comportement de la variable dépendante de notre
systhéme. La quantité du vin sans IG produit par département semble
pouvoir être correlé à partir de la figure suivante.

\FloatBarrier

\begin{figure}[!htbp]

{\centering \includegraphics{note2pres_files/figure-latex/unnamed-chunk-18-1} 

}

\caption{Les quantité du vin non-IG moyennes par département}\label{fig:unnamed-chunk-18}
\end{figure}

\FloatBarrier

Puis, nous observons le comportement du reste des variables (les
representations graphiques sont grouppés dans l'annexe X). L'indice des
prix se comporte pratiquement comme quantité du vin produite, car cet
indice fut construit par biais de cette variable. Les autres moyennes ne
semblent pas avoir des structures corrélés dans l'espace au niveau de la
France. Dans notre analyse nous nous laissons liberté d'ignorer les
effets possibles d'autocorrelation spatiale dans nos données, parce que
au moment de constructions de notre base des données nous avons ignoré
les département ne produisant pas le vin simple, mais qui peuvent quand
même jouer son rôle si nous étions à prendre en compte la structure
spatiale des nos données.

\hypertarget{etude-de-la-variance}{%
\subsection{Etude de la variance}\label{etude-de-la-variance}}

Passons maintenant à l'étude de la variance. Nous allons décortiquer la
variance par type (between et within) afin d'obtenir une idée sur le
choix preferable de la dimention d'aggregation des nos données, car il
peut se reveler que la théorie ne corresponde pas à la réalitée (ex:
nous faisons face aux effets fixes par année et non par département).

Le tableau suivant regrouppe les statistiques déscriptives essentielles
:

\begin{itemize}
\tightlist
\item
  Moyennes
\item
  Variance sur l'échantillon complet
\item
  Variance \emph{between}
\item
  Variance \emph{within}
\end{itemize}

\FloatBarrier

\begin{table}[!htbp] \centering 
  \caption{Etude de la variance} 
  \label{} 
\begin{tabular}{@{\extracolsep{5pt}} ccccc} 
\\[-1.8ex]\hline 
\hline \\[-1.8ex] 
 & Mean & Overall & Between & Within \\ 
\hline \\[-1.8ex] 
Index prix & $1.431$ & $1.339$ & $1.012$ & $0.883$ \\ 
Index pesticides & $1.257$ & $0.483$ & $0.335$ & $0.350$ \\ 
Surface & $4.892$ & $1.986$ & $1.955$ & $0.410$ \\ 
Revenus & $9.891$ & $0.061$ & $0.061$ & $0.011$ \\ 
Temps & $3$ & $1.416$ & $0$ & $1.416$ \\ 
\hline \\[-1.8ex] 
\end{tabular} 
\end{table}

\FloatBarrier

Il est facile a rémarquer que la variance \emph{between} est plus
significative que la variance \emph{within}. Cela nous amêne à l'idée
qu'il faut utiliser un modèle qui permettra estimer et corriger ces
inégalités entre les individus, car nous sommes plus interessés par des
effets individuels moyens (les effets moyens pour tous les
départemetns). Ce qui est completement conforme à notre hypothèse qu'on
a exprimé lors de la formalisation du modèle économique théorique.

De plus, il est interessant d'observer les résultats obtenus pour le
test de Chow comparant le modèle complet (\emph{pooled model}) contre
les modèles au effet fixes et randomes. Le tableau suivant régrouppe les
p-valeurs de ce test pour les modèles univariées differents.

\FloatBarrier

\begin{table}[!htbp] \centering 
  \caption{Les p-valeurs de pooling-test de Chow} 
  \label{} 
\begin{tabular}{@{\extracolsep{5pt}} ccc} 
\\[-1.8ex]\hline 
\hline \\[-1.8ex] 
 & Random & Fixed \\ 
\hline \\[-1.8ex] 
Index prix & $0$ & $0$ \\ 
Index pesticides & $0.354$ & $0.294$ \\ 
Surface & $0$ & $0.0001$ \\ 
Revenus & $0.297$ & $0.247$ \\ 
\hline \\[-1.8ex] 
\end{tabular} 
\end{table}

\FloatBarrier

Sauf le cas de la surface nous ne pouvons pas rejeter l'hypothese nulle,
specifiant que les individus ont des effets identiques pour toute la
population.

\hypertarget{letude-des-types-deffets}{%
\subsection{L'étude des types d'effets}\label{letude-des-types-deffets}}

Nous avons déjà vu, qu'il est fortement probable que nous faisons face à
un modèle aux effets fixes individuelles. Il faut quand même le
justifier. Pour faire cela, nous allons effectuer le test de
multiplicateur de Lagrange sur la nature des effets (individuels,
temporels ou en double dimention). Selon les résultats des tests il est
difficile de choisir arbitrairement un type des effets. Il est évidente
que nous avons des effets fixes au niveau individuel ou des fixes en
double dimention pour toutes les variables.

\FloatBarrier

\begin{table}[!htbp] \centering 
  \caption{p-valeurs de Lagrange multiplier test} 
  \label{} 
\begin{tabular}{@{\extracolsep{5pt}} cccc} 
\\[-1.8ex]\hline 
\hline \\[-1.8ex] 
 & Individual & Time & Twoways \\ 
\hline \\[-1.8ex] 
Index prix & $0$ & $0.256$ & $0$ \\ 
Index pesticides & $0$ & $0.229$ & $0$ \\ 
Surface & $0$ & $0.030$ & $0$ \\ 
Revenus & $0$ & $0.248$ & $0$ \\ 
\hline \\[-1.8ex] 
\end{tabular} 
\end{table}

\FloatBarrier

Selon les résultats obtenus, ainsi que les evidences théoriques des
études anterieurs nous décidons de ne garder que les effets fixes au
niveau individuel afin de faciliter l'analyse.

\hypertarget{lanalyse-de-la-correlation}{%
\subsection{L'analyse de la
correlation}\label{lanalyse-de-la-correlation}}

Dans le tableau ci-dessous nous presentons les correlations des
variables après la correction pour les effets fixes individuels (nous
effectuons la transformation \emph{within} sur nos données en
substrayant les moyennes individuelles pour l'ensemble des variables).
Dans les annexes nous proposons egalement un tableau de correlation pour
les données non-transformées, ce qui permet d'observer les inégalités et
une pauvre répresentativitée des liens entres les variables pour les
données initiales.

Particulierement nous pouvons remarquer une forte correlation entre la
quantité offerte et le prix d'équilibre. Egalement \ldots{}

\hypertarget{modelisation}{%
\section{6. Modèlisation}\label{modelisation}}

\noindent

\rule[0.5ex]{\linewidth}{1pt}

\textcolor{red}{Séparer les modèles (OLS, 3SLS avec justification par 2SLS et la comparaison avec i3SLS, clusters en OLS et 3SLS).}

\textcolor{red}{Justifier le choix des modèles par 3 cas théoriques. Discuter les avantages et les inconveniences}

\textcolor{red}{Ajouter des liens avec des études méthodologiques precedents.}

\textcolor{red}{Pour le modèle 2SLS préciser la forme, tester les instruments}

\textcolor{red}{Arbitrage du choix de 2SLS vs 3SLS}

\noindent

\rule[0.5ex]{\linewidth}{1pt}

Cette partie du travail abordera la formulation économétrique du notre
problème. Nous allons débuter par la présentation des notions théoriques
implimentés dans ce travail, suivis par la formalisation économétrique
du modèle théorique que nous avons spécifié dans la séction 5. Après,
nous expliquerons la stratégie d'identification utilisée.

\hypertarget{presentation-de-la-methodologie}{%
\subsection{Presentation de la
méthodologie}\label{presentation-de-la-methodologie}}

L'AIDS (\emph{almost ideal demand system}) et les autres modèles de
demande cités dans la littérature ont de nombreuses lacunes qui les
rendent impropres pour l'estimation du marché du vin, selon Cembalo,
Caracciolo, and Pomarici (2014). Quand même, dans notre étude nous
allons utiliser un approche similaire à ce modèle là, sous des
suppositions restrictives.

Ce modèle nous permettra de simuler l'équilibre sur le marché du vin,
prenant ainsi en compte la pluspart des facteurs incitant les
producteurs du vin d'utiliser les pésticides.

\hypertarget{modele-econometrique}{%
\subsection{Modèle économétrique}\label{modele-econometrique}}

Dans cette séction nous allons presenter une par une nos modèles
économétriques correspondant chacune à un des trois cadres théoriques
possibles. Tous les modèles visent à estimer les effets moyenns pour
tous les départements sous des hypothèses differentes du fonctionnement
du marché. Dans tous les cas, l'aggregation des effets au niveau
national (ou au niveau des grouppes) nous permet de mitiger les biais
eventuels, liés à la misspecification du modèle.

Pour le cadre où nous n'observons pas des interactions entre la demande
et l'offre sur le marché (M1), nous ésimons un modèle simple. Nous
écrivons notre modèle sous la forme suivante :

\begin{equation*}
  qo_{i,t} = a_1 + b Po_{i,t} + c X_{i,t} + u_{i,t}
\end{equation*}

A ce point nous avons un choix : soit nous supposons que les
agriculteurs sont des preneurs des prix, ce qui nous permet de traiter
le prix comme une variable exogène; soit nous dévrions construire un
estimateur IV afin de traiter l'endogénéité eventuelle de l'index des
prix. Evidement le premier cas est le plus simple, mais pour justifier
l'implementation de cette méthode nous dévrions effectuer des tests
d'énogénéité des prix. Le deuxième cas est beaucoup plus réaliste,
puisque les viticulteurs sont rarement preneurs des prix et l'offre
aussi joue son rôle sur l'équilibre du marché.

Dans la dérnier situation nous utilisons les idées de MacKay and Miller
(2018), supposant que les variables détérminant la demande sont des
instruments fiables pour la prédiction des variables endogènes dans
l'équation d'offre (bien que dans notre cas nous ignorons les effets des
intéractions entre l'offre et la demande). Particulieremnt ici nous
pourrions utiliser les données sur les revenus afin d'instrumenter le
niveau des prix (l'indice des prix du vin).

Passons maintenant au modèle plus complexe (M2), basé sur l'hypothèse
que la demande influence l'offre, affectant également le mode
d'utilisation des pésticides par les agriculteurs. Nous pouvons réécrire
notre système d'equations dans ce cas sous la forme suivante :

\begin{align*}
  qo_{i,t} & = a_1 + b Po_{i,t} + c X_{i,t} + u_{i,t} \\ 
  qd_{i,t} & = \alpha_{i} + \beta Pd_{i,t} + \gamma Z_{i,t} + \epsilon_{i,t}  \\
\end{align*}

Nous posons que l'offre et la demande sont egaux au niveau de
département : \(qd_{i,t} = qo_{i,t}\). C'est à dire l'offre interne du
département vise à satisfaire la demande interne du même département.

En termes d'aggregation ex-post des effets estimés, nous sommes sensé de
tomber sur l'équilibre au niveau du marché national. En d'autre mots, le
système (qui implique : \(Qd = Qo\)) :

\begin{equation*}
  qd_{i,t} = qo_{i,t}
\end{equation*}

Au point d'équilibre nous rencontrons également l'égalité des prix :

\begin{equation*}
  Po_{1,t} = Pd_{1,t}
\end{equation*}

De cette façon nous obtenons un système des systèmes des équations. En
simplifiant l'écriture nous pouvons la representer sous la forme
suivante :

\begin{align*}
  q_{i,t} & = \alpha_{i} + \beta P_{i,t} + \gamma Z_{i,t} + \epsilon_{i,t} \\
  q_{i,t} & = a_i + b P_{i,t} + c X_{i,t} + u_{i,t}
\end{align*}

Et finalement, nous pouvons estimer les deux modèles (M1 et M2) en
regrouppant les département par leurs caractéristiques. Appelons ces
modèles M3.1 et M3.2 réspectivement.

Le prémiér prénant la forme :

\begin{align*}
  qo_{i,t} & = a_1 + b Po_{i,t} + c X_{i,t} + u_{i,t} \\ 
\end{align*}

Tandis que le dérnier :

\begin{align*}
  q_{i_{c},t} & = \alpha_{i_{c}} + \beta P_{i_{c},t} + \gamma Z_{i_{c},t} + \epsilon_{i_{c},t} \\
  q_{i_{c},t} & = a_i + b P_{i_{c},t} + c X_{i_{c},t} + u_{i_{c},t}
\end{align*}

Avec \(c\) décrivant l'appartenance de département à un des clusters.

Pour finir cette partie, resumons que nous avons à notre disposition
plusieurs chemins differents à traiter ce modèle du point de vue
économétrique. Le plus simple est d'estimer les effets des pésticides
sur l'offre du vin en ignorant les impacts du comportement des
consommaterus sur les producteurs. Cette méthode implique une éstimation
par OLS simples (ou IV-OLS, lesquels introduisent la notion
d'éndogénéité des prix). D'autre coté, nous pouvons implementer les
tripples moindre carrés (nous dévrions comparer les résultats obtenus
avec un système des équations non-réliées, éstimé par 2SLS afin de
traiter l'éndogenèité), qui nous permettrons d'obtenir des résultats
identiques aux résultats d'estimations des équation structurelles sous
l'hypothèse des intéractions entre l'offre et la demande. Cette méthode
offre la possibilité d'estimer le système d'équations avec plusieurs
variables endogèenes en prenant en compte les deux coté du marché à la
fois. Finalement, si on trouve qu'il y existe une heterogeneité entre
les départements en termes d'équilibre interne, nous pourrions réestimer
les modèles en clusterisant nos \emph{individus} (départements) par des
differents classes selon leurs attributs, pour après estimer les
equations par cluster.

\hypertarget{hypotheses-sur-les-resultats}{%
\subsection{Hypothèses sur les
résultats}\label{hypotheses-sur-les-resultats}}

Nous attendons à ce que l'éstimateur de 3SLS, qui permet de capter les
effets de correlations entre les équation en presence de plusieures
variables exogènes nous permettra d'obtenir des estimations les plus
fiables. Cette méthode nous permet à depasser le biais de simultanéité
qui apparaisse dans le cas d'estimation des systèmes d'équations liées
(dans notre cas nous étudions les effets des pésticides sur l'offre et
production du vin simple sous hypothèse de présence des effets du
marché). L'estimateur pareil donne des résultats similaires à
l'éstimateur de ILS (\emph{indirect least squares}). De plus, sa version
iterée (qui converge à des résultats similaires à ceux obtenus par
l'éstimation avec maximum de vraisamblance) donne des résultats avec la
moindre variance.

Les propriétés de cet éstimateurs sont :

\begin{itemize}
\tightlist
\item
  Consistence ;
\item
  Efficience (asymptotique) ;
\item
  La distribuitions pour les estimateurs suit une loi normale suelement
  dans des grands échantillons.
\end{itemize}

Quand même dès le debut nous envisageons que cet éstimateur ne refletera
pas la nature du marché. C'est pourquoi nous, dans ce travail, testons
plusieurs modèles.

Parmis les inconveniences eventuelles on a également la faible
representation des effets hetérogenes entre les départements par le
modèle. Nous estimons seulemnt les effets moyens et ainsi ignorons les
differences des élasticités pour des départements differents.
Hereusement ce problème peut être rémédiée par l'introduction des
clusters, regrouppant des département ayant le comportement similaire.

Finalement, il existe des effets qu'on ignore completement, mais qui
risquent d'intervenir. Par example, nous ignorons la présence
d'autocorrelation spatiale et/ou temporelle dans notre modèle.
Egalement, un nombre probablement insuffisant des facteurs est utilisé
dans ce modèle, ce qui risaue d'apporter le biais des variables omises
dans nos estimations.

\hypertarget{resultats-des-estimations}{%
\section{7. Résultats des estimations}\label{resultats-des-estimations}}

Dans cette séction nous allons presenter les résultats économétriques
pour des differents modèles ainsi que les comparer.

Nous estimons un enseble des differents modèles possibles afin de
pouvoir choisir la méthode la plus raisonnable. Les modèles suivantes
sont traitées séparement :

\begin{itemize}
\tightlist
\item
  M1 : modèle simple sans intéractions entre l'offre et la demande ;
\item
  M2 : modèle complexe visant à integrer les intéractions entre l'offre
  et la demande en presence des variables éndogènes ;
\item
  M3 : les modèles sur les données clustérisés (M3.1 et M3.2
  réspectivement pour les deux cas précedents).
\end{itemize}

\hypertarget{les-resultats-en-absence-des-interactions}{%
\subsection{Les résultats en absence des
intéractions}\label{les-resultats-en-absence-des-interactions}}

\FloatBarrier

\begin{table}[!htbp]
\begin{center}
\begin{tabular}{l c c c }
\hline
 & OLS & WLS & SUR \\
\hline
Demande: ipi        & $0.33^{***}$  & $0.33^{***}$  & $0.32^{***}$ \\
                    & $(0.02)$      & $(0.02)$      & $(0.02)$     \\
Demande: ri         & $-9.00^{***}$ & $-9.00^{***}$ & $-1.10$      \\
                    & $(1.46)$      & $(1.46)$      & $(0.66)$     \\
Offre: ipi          & $0.30^{***}$  & $0.30^{***}$  & $0.32^{***}$ \\
                    & $(0.02)$      & $(0.02)$      & $(0.02)$     \\
Offre: si           & $0.23^{***}$  & $0.23^{***}$  & $0.03$       \\
                    & $(0.04)$      & $(0.04)$      & $(0.02)$     \\
Offre: iki          & $-0.16^{***}$ & $-0.16^{***}$ & $-0.02$      \\
                    & $(0.05)$      & $(0.05)$      & $(0.02)$     \\
\hline
Demande: R$^2$      & 0.50          & 0.50          & 0.46         \\
Offre: R$^2$        & 0.52          & 0.52          & 0.46         \\
Demande: Adj. R$^2$ & 0.50          & 0.50          & 0.45         \\
Offre: Adj. R$^2$   & 0.52          & 0.52          & 0.46         \\
Num. obs. (total)   & 690           & 690           & 690          \\
\hline
\multicolumn{4}{l}{\scriptsize{$^{***}p<0.001$, $^{**}p<0.01$, $^*p<0.05$}}
\end{tabular}
\caption{Statistical models}
\label{table : ols, wls and sur}
\end{center}
\end{table}

\FloatBarrier

\hypertarget{les-resultats-2sls-w2sls-3sls-et-i3sls}{%
\subsection{Les résultats 2SLS, W2SLS, 3SLS et
i3SLS}\label{les-resultats-2sls-w2sls-3sls-et-i3sls}}

\FloatBarrier

\FloatBarrier

\begin{table}[!htbp]
\begin{center}
\begin{tabular}{l c c c c }
\hline
 & 2SLS & W2SLS & 3SLS & i3SLS \\
\hline
Demande: ipi        & $0.79^{***}$   & $0.79^{***}$   & $0.79^{***}$   & $0.79^{***}$   \\
                    & $(0.15)$       & $(0.15)$       & $(0.15)$       & $(0.15)$       \\
Demande: ri         & $-13.07^{***}$ & $-13.07^{***}$ & $-13.07^{***}$ & $-13.07^{***}$ \\
                    & $(2.76)$       & $(2.76)$       & $(2.76)$       & $(2.76)$       \\
Offre: ipi          & $-0.28$        & $-0.28$        & $-0.25$        & $-0.25$        \\
                    & $(0.25)$       & $(0.25)$       & $(0.25)$       & $(0.24)$       \\
Offre: si           & $0.47^{***}$   & $0.47^{***}$   & $0.45^{***}$   & $0.45^{***}$   \\
                    & $(0.13)$       & $(0.13)$       & $(0.13)$       & $(0.12)$       \\
Offre: iki          & $-0.11$        & $-0.11$        & $-0.17^{*}$    & $-0.17^{*}$    \\
                    & $(0.09)$       & $(0.09)$       & $(0.08)$       & $(0.08)$       \\
\hline
Demande: R$^2$      & -0.41          & -0.41          & -0.41          & -0.41          \\
Offre: R$^2$        & -0.87          & -0.87          & -0.74          & -0.75          \\
Demande: Adj. R$^2$ & -0.42          & -0.42          & -0.42          & -0.42          \\
Offre: Adj. R$^2$   & -0.89          & -0.89          & -0.75          & -0.76          \\
Num. obs. (total)   & 690            & 690            & 690            & 690            \\
\hline
\multicolumn{5}{l}{\scriptsize{$^{***}p<0.001$, $^{**}p<0.01$, $^*p<0.05$}}
\end{tabular}
\caption{Statistical models}
\label{table : 2sls, w2sls, 3sls and fiml}
\end{center}
\end{table}

\FloatBarrier

\hypertarget{clusterisation-et-modelisation-par-groupe}{%
\subsection{Clusterisation et modèlisation par
groupe}\label{clusterisation-et-modelisation-par-groupe}}

\hypertarget{between}{%
\subsubsection{\texorpdfstring{\emph{Between}}{Between}}\label{between}}

Nous avons vus dans le comportement des résidus une nature non-aléatoire
grouppé. Cela nous amène à l'idée de construire k-clusters pour
modèliser les rélations par grouppe.

Nous supposons que les départements ayant des valeurs moyennes
interannuelles proches (transformation Between) ont le comportement
identique. La clusterisation est effectué sur les données Between pour
les départements.

Nous povons supposer que le nombre des clusters optimal est entre 3 et
5. Prenant en compte les graphiques des résidus vus lros d'analse des
modèles nous allons supposer qu'il n'y a que 3 clusters principaux.

\FloatBarrier

\begin{figure}[!htbp]

{\centering \includegraphics{note2pres_files/figure-latex/unnamed-chunk-37-1} 

}

\caption{Le choix des clusters}\label{fig:unnamed-chunk-37}
\end{figure}

\FloatBarrier

\hypertarget{within}{%
\subsubsection{\texorpdfstring{\emph{Within}}{Within}}\label{within}}

Nous avons vus dans le comportement des résidus une nature non-aléatoire
grouppé. Cela nous amène à l'idée de construire k-clusters pour
modèliser les rélations par grouppe.

D'abord on compare le comportement des cluster pour les données à
l'information complete et les données Within.

Comme nous pouvons voir dans les résultats le nombre des cluster
optimaux est trop large pour les séparer dans l'analyse :

Nous povons supposer que le nombre des clusters optimal est entre 6 et
15.

\FloatBarrier

\begin{center}\includegraphics{note2pres_files/figure-latex/unnamed-chunk-43-1} \end{center}

\FloatBarrier

\hypertarget{information-complete}{%
\subsubsection{Information complete}\label{information-complete}}

Dans le cas d'information complete on a :

Nous povons supposer que le nombre des clusters optimal est entre 3 et
5. Prenant en compte les graphiques des résidus vus lros d'analse des
modèles nous allons supposer qu'il n'y a que 3 clusters principaux.

\FloatBarrier

\begin{center}\includegraphics{note2pres_files/figure-latex/unnamed-chunk-48-1} \end{center}

\FloatBarrier

\hypertarget{modelisation-1}{%
\subsubsection{Modèlisation}\label{modelisation-1}}

\FloatBarrier

Nous evaluons le système en introduisant les variables de grouppe (dummy
variables) sous l'hypothèse des résidus joints.

Les résultats obtenus sont suivants :

\FloatBarrier

\begin{table}[!htbp]
\begin{center}
\begin{tabular}{l c c c }
\hline
 & OLS & 2SLS & 3SLS \\
\hline
Demande: ipi        & $0.33^{***}$   & $0.15^{***}$   & $0.16^{***}$ \\
                    & $(0.02)$       & $(0.03)$       & $(0.03)$     \\
Demande: ri1        & $-1.69$        & $0.99$         & $0.48$       \\
                    & $(1.92)$       & $(2.17)$       & $(1.19)$     \\
Demande: ri2        & $-17.14^{***}$ & $-18.38^{***}$ & $-3.30^{**}$ \\
                    & $(2.09)$       & $(2.35)$       & $(1.26)$     \\
Demande: ri3        & $-22.12^{*}$   & $13.13$        & $13.93$      \\
                    & $(10.47)$      & $(12.23)$      & $(12.17)$    \\
Offre: ipi          & $0.37^{***}$   & $0.16^{***}$   & $0.19^{***}$ \\
                    & $(0.02)$       & $(0.03)$       & $(0.03)$     \\
Offre: si1          & $0.14$         & $0.21^{*}$     & $0.03$       \\
                    & $(0.08)$       & $(0.09)$       & $(0.05)$     \\
Offre: si2          & $0.26^{***}$   & $0.31^{***}$   & $0.06^{*}$   \\
                    & $(0.04)$       & $(0.05)$       & $(0.03)$     \\
Offre: si3          & $-2.58^{***}$  & $-0.68$        & $-0.58$      \\
                    & $(0.37)$       & $(0.47)$       & $(0.46)$     \\
Offre: iki1         & $-0.06$        & $0.00$         & $-0.01$      \\
                    & $(0.08)$       & $(0.09)$       & $(0.05)$     \\
Offre: iki2         & $-0.18^{***}$  & $-0.20^{***}$  & $-0.04$      \\
                    & $(0.05)$       & $(0.06)$       & $(0.03)$     \\
Offre: iki3         & $13.81^{***}$  & $8.01^{*}$     & $4.03$       \\
                    & $(3.21)$       & $(3.74)$       & $(2.99)$     \\
\hline
Demande: R$^2$      & 0.54           & 0.42           & 0.37         \\
Offre: R$^2$        & 0.59           & 0.46           & 0.42         \\
Demande: Adj. R$^2$ & 0.54           & 0.42           & 0.36         \\
Offre: Adj. R$^2$   & 0.59           & 0.45           & 0.41         \\
Num. obs. (total)   & 690            & 690            & 690          \\
\hline
\multicolumn{4}{l}{\scriptsize{$^{***}p<0.001$, $^{**}p<0.01$, $^*p<0.05$}}
\end{tabular}
\caption{Statistical models}
\label{table : ols, 2sls et 3sls, full information clusters}
\end{center}
\end{table}

\FloatBarrier

\hypertarget{conclusions}{%
\section{9. Conclusions}\label{conclusions}}

\begin{itemize}
\tightlist
\item
  Le marché du vin
\item
  Le rôle des pésticides\\
\item
  Validité
\end{itemize}

\FloatBarrier

\hypertarget{le-marche-du-vin}{%
\subsection{Le marché du vin}\label{le-marche-du-vin}}

\begin{itemize}
\tightlist
\item
  Un comportement inattendus

  \begin{itemize}
  \item
    Les effets de substitution contre les produits de la haute gamme
  \item
    Les effets négatives du revenu
  \item
  \end{itemize}
\end{itemize}

\FloatBarrier

\hypertarget{le-role-des-pesticides}{%
\subsection{Le rôle des pésticides}\label{le-role-des-pesticides}}

\begin{itemize}
\tightlist
\item
  Confirmation des résultats des études précedentes

  \begin{itemize}
  \tightlist
  \item
    Utilisés pour réduire les pertes
  \end{itemize}
\end{itemize}

\FloatBarrier

\hypertarget{validite}{%
\subsection{Validité}\label{validite}}

\begin{itemize}
\tightlist
\item
  Faible validité du modèle économétrique

  \begin{itemize}
  \tightlist
  \item
    Variables ommises
  \end{itemize}
\end{itemize}

\FloatBarrier

\newpage

\hypertarget{annexes}{%
\section{Annexes}\label{annexes}}

\hypertarget{a-les-statistiques-descriptives}{%
\subsection{A Les statistiques
déscriptives}\label{a-les-statistiques-descriptives}}

\hypertarget{a1-les-moyennes-par-departement}{%
\subsubsection{A1 Les moyennes par
département}\label{a1-les-moyennes-par-departement}}

\FloatBarrier

\begin{figure}[!htbp]

{\centering \includegraphics{note2pres_files/figure-latex/unnamed-chunk-56-1} 

}

\caption{Les valeurs moyennes par département, partie 1}\label{fig:unnamed-chunk-56}
\end{figure}

\FloatBarrier

\FloatBarrier

\begin{figure}[!htbp]

{\centering \includegraphics{note2pres_files/figure-latex/unnamed-chunk-57-1} 

}

\caption{Les valeurs moyennes par département, partie 2}\label{fig:unnamed-chunk-57}
\end{figure}

\FloatBarrier

\newpage

\hypertarget{a2-les-graphiques-bivaries}{%
\subsubsection{A2 Les graphiques
bivariés}\label{a2-les-graphiques-bivaries}}

\hypertarget{cas-general}{%
\paragraph{Cas général}\label{cas-general}}

\FloatBarrier

\begin{figure}[!htbp]

{\centering \includegraphics{note2pres_files/figure-latex/unnamed-chunk-58-1} 

}

\caption{L'étude bivarié, partie 1}\label{fig:unnamed-chunk-58}
\end{figure}

\FloatBarrier

\FloatBarrier

\begin{figure}[!htbp]

{\centering \includegraphics{note2pres_files/figure-latex/unnamed-chunk-59-1} 

}

\caption{L'étude bivarié, partie 2}\label{fig:unnamed-chunk-59}
\end{figure}

\FloatBarrier

\hypertarget{transformation-within}{%
\paragraph{\texorpdfstring{Transformation
\emph{Within}}{Transformation Within}}\label{transformation-within}}

\FloatBarrier

\begin{figure}[!htbp]

{\centering \includegraphics{note2pres_files/figure-latex/unnamed-chunk-60-1} 

}

\caption{Rélations bivariés dans le cas de transformation within, partie 1}\label{fig:unnamed-chunk-60}
\end{figure}

\FloatBarrier

\begin{figure}[!htbp]

{\centering \includegraphics{note2pres_files/figure-latex/unnamed-chunk-61-1} 

}

\caption{Rélations bivariés dans le cas de transformation within, partie 2}\label{fig:unnamed-chunk-61}
\end{figure}

\FloatBarrier

\newpage

\hypertarget{a3-la-correlation}{%
\subsubsection{A3 La correlation}\label{a3-la-correlation}}

\hypertarget{cas-general-1}{%
\paragraph{Cas général}\label{cas-general-1}}

Le premier tableau combrend les résultats pour les données
telles-quelles, le deuxieme par contre integre les résultats pour les
données sous la trasformation \emph{within}.

\FloatBarrier

\begin{longtable}[]{@{}lrrrrrr@{}}
\toprule
& Quantité du vin & IP & Surface & Revenus & Index pésticides &
Temps\tabularnewline
\midrule
\endhead
Quantité du vin & 1.0000 & 0.0177 & 0.9559 & -0.0266 & -0.0667 &
-0.0360\tabularnewline
IP & 0.0177 & 1.0000 & -0.0513 & 0.0065 & -0.0590 &
0.1082\tabularnewline
Surface & 0.9559 & -0.0513 & 1.0000 & -0.0567 & -0.0486 &
-0.0640\tabularnewline
Revenus & -0.0266 & 0.0065 & -0.0567 & 1.0000 & -0.0433 &
0.1188\tabularnewline
Index pésticides & -0.0667 & -0.0590 & -0.0486 & -0.0433 & 1.0000 &
0.2971\tabularnewline
Temps & -0.0360 & 0.1082 & -0.0640 & 0.1188 & 0.2971 &
1.0000\tabularnewline
\bottomrule
\end{longtable}

\FloatBarrier

\hypertarget{transformation-within-1}{%
\paragraph{\texorpdfstring{Transformation
\emph{Within}}{Transformation Within}}\label{transformation-within-1}}

Les rélations entre les variables mieux ressortent pour les données
transformées.

\FloatBarrier

\begin{longtable}[]{@{}lrrrrrr@{}}
\toprule
& Quantité du vin & IP & Surface & Revenus & Index pésticides &
Temps\tabularnewline
\midrule
\endhead
Quantité du vin & 1.0000 & 0.6656 & 0.3655 & -0.1601 & -0.1813 &
-0.1994\tabularnewline
IP & 0.6656 & 1.0000 & 0.1862 & 0.1119 & -0.0108 & 0.1640\tabularnewline
Surface & 0.3655 & 0.1862 & 1.0000 & -0.1657 & -0.2035 &
-0.3103\tabularnewline
Revenus & -0.1601 & 0.1119 & -0.1657 & 1.0000 & 0.2103 &
0.6522\tabularnewline
Index pésticides & -0.1813 & -0.0108 & -0.2035 & 0.2103 & 1.0000 &
0.4100\tabularnewline
Temps & -0.1994 & 0.1640 & -0.3103 & 0.6522 & 0.4100 &
1.0000\tabularnewline
\bottomrule
\end{longtable}

\FloatBarrier

\newpage

\hypertarget{b-analyse-des-resultats-ols-wls-et-sur}{%
\subsection{B Analyse des résultats OLS, WLS et
SUR}\label{b-analyse-des-resultats-ols-wls-et-sur}}

\hypertarget{b1-independance-des-residus}{%
\subsubsection{B1 Independance des
résidus}\label{b1-independance-des-residus}}

\FloatBarrier

\begin{longtable}[]{@{}lrrrrrr@{}}
\toprule
& OLS D & OLS O & WLS D & WLS O & SUR D & SUR O\tabularnewline
\midrule
\endhead
Vin & 0.7080 & 0.6932 & 0.7080 & 0.6932 & 0.7458 & 0.7454\tabularnewline
IP & 0.0000 & 0.0000 & 0.0000 & 0.0000 & 0.0000 & 0.0000\tabularnewline
Surface & 0.2786 & 0.0000 & 0.2786 & 0.0000 & 0.3202 &
0.2841\tabularnewline
Revenus & 0.0000 & -0.2389 & 0.0000 & -0.2389 & -0.2793 &
-0.3067\tabularnewline
Pesticides & -0.1749 & 0.0000 & -0.1749 & 0.0000 & -0.2277 &
-0.2021\tabularnewline
\bottomrule
\end{longtable}

\FloatBarrier

\FloatBarrier

\begin{figure}[!htbp]

{\centering \includegraphics{note2pres_files/figure-latex/unnamed-chunk-70-1} 

}

\caption{Les résidus contre la variable prédite}\label{fig:unnamed-chunk-70}
\end{figure}

\FloatBarrier

\hypertarget{b2-lautocorrelation}{%
\subsubsection{B2 L'autocorrelation}\label{b2-lautocorrelation}}

\FloatBarrier

\begin{table}[!htbp] \centering 
  \caption{Les statistiques test de Durbin-Watson} 
  \label{} 
\begin{tabular}{@{\extracolsep{5pt}} cccc} 
\\[-1.8ex]\hline 
\hline \\[-1.8ex] 
 & OLS & WLS & SUR \\ 
\hline \\[-1.8ex] 
Equation de demande & $0.653$ & $0.653$ & $0.687$ \\ 
Equation d'offre & $0.627$ & $0.627$ & $0.683$ \\ 
\hline \\[-1.8ex] 
\end{tabular} 
\end{table}

\FloatBarrier

\hypertarget{b3-test-de-lheteroskedacite}{%
\subsubsection{B3 Test de
l'hétéroskedacité}\label{b3-test-de-lheteroskedacite}}

\FloatBarrier

\begin{table}[!htbp] \centering 
  \caption{Les résultat du test de Bartlett sur l'heteroscedacité} 
  \label{} 
\begin{tabular}{@{\extracolsep{5pt}} cccc} 
\\[-1.8ex]\hline 
\hline \\[-1.8ex] 
 & OLS & WLS & SUR \\ 
\hline \\[-1.8ex] 
Equation de demande & $0$ & $0$ & $0$ \\ 
Equation d'offre & $0$ & $0$ & $0$ \\ 
\hline \\[-1.8ex] 
\end{tabular} 
\end{table}

\FloatBarrier

\newpage

\hypertarget{b4-la-normalite-des-residus}{%
\subsubsection{B4 La normalité des
résidus}\label{b4-la-normalite-des-residus}}

\FloatBarrier

\FloatBarrier

\begin{table}[!htbp] \centering 
  \caption{Shapiro-Wilk test de normalité des résidus} 
  \label{} 
\begin{tabular}{@{\extracolsep{5pt}} cccc} 
\\[-1.8ex]\hline 
\hline \\[-1.8ex] 
 & OLS & WLS & SUR \\ 
\hline \\[-1.8ex] 
Equation de demande & $0$ & $0$ & $0$ \\ 
Equation d'offre & $0$ & $0$ & $0$ \\ 
\hline \\[-1.8ex] 
\end{tabular} 
\end{table}

\FloatBarrier

\FloatBarrier

\begin{figure}[!htbp]

{\centering \includegraphics{note2pres_files/figure-latex/unnamed-chunk-77-1} 

}

\caption{Les PDF des résidus}\label{fig:unnamed-chunk-77}
\end{figure}

\FloatBarrier

\newpage

\hypertarget{c-analyse-des-resultats-2sls-w2sls-3sls-et-i3sls}{%
\subsection{C Analyse des résultats 2SLS, W2SLS, 3SLS et
i3SLS}\label{c-analyse-des-resultats-2sls-w2sls-3sls-et-i3sls}}

\FloatBarrier

\FloatBarrier

\hypertarget{c1-independance-des-residus}{%
\subsubsection{C1 Independance des
résidus}\label{c1-independance-des-residus}}

\FloatBarrier

\FloatBarrier

\FloatBarrier

\begin{figure}[!htbp]

{\centering \includegraphics{note2pres_files/figure-latex/unnamed-chunk-82-1} 

}

\caption{Les résidus contre la variable prédite}\label{fig:unnamed-chunk-82}
\end{figure}

\FloatBarrier

\FloatBarrier

\begin{figure}[!htbp]

{\centering \includegraphics{note2pres_files/figure-latex/unnamed-chunk-83-1} 

}

\caption{Les résidus et les prédictions, le cas de i3SLS}\label{fig:unnamed-chunk-83}
\end{figure}

\FloatBarrier

\newpage

\hypertarget{c2-lautocorrelation}{%
\subsubsection{C2 L'autocorrelation}\label{c2-lautocorrelation}}

\FloatBarrier

\begin{table}[!htbp] \centering 
  \caption{Les resultats du test de Durbin-Watson} 
  \label{} 
\begin{tabular}{@{\extracolsep{5pt}} cccc} 
\\[-1.8ex]\hline 
\hline \\[-1.8ex] 
 & 2SLS & 3SLS & i3SLS \\ 
\hline \\[-1.8ex] 
Equation de demande & $0.618$ & $0.618$ & $0.618$ \\ 
Equation d'offre & $0.637$ & $0.638$ & $0.638$ \\ 
\hline \\[-1.8ex] 
\end{tabular} 
\end{table}

\FloatBarrier

\hypertarget{c3-test-de-lheteroskedacite}{%
\subsubsection{C3 Test de
l'hétéroskedacité}\label{c3-test-de-lheteroskedacite}}

\FloatBarrier

\% Table created by stargazer v.5.2.2 by Marek Hlavac, Harvard
University. E-mail: hlavac at fas.harvard.edu \% Date and time: lun.,
déc. 23, 2019 - 20:15:19

\begin{table}[!htbp] \centering 
  \caption{Test de Bartlett sur l'heterockedacité} 
  \label{} 
\begin{tabular}{@{\extracolsep{5pt}} cccc} 
\\[-1.8ex]\hline 
\hline \\[-1.8ex] 
 & 2SLS & 3SLS & i3SLS \\ 
\hline \\[-1.8ex] 
Equation de demande & $0$ & $0$ & $0$ \\ 
Equation d'offre & $0$ & $0$ & $0$ \\ 
\hline \\[-1.8ex] 
\end{tabular} 
\end{table}

\FloatBarrier

\newpage

\hypertarget{c4-la-normalite-des-residus}{%
\subsubsection{C4 La normalité des
résidus}\label{c4-la-normalite-des-residus}}

\FloatBarrier

\FloatBarrier

\begin{table}[!htbp] \centering 
  \caption{Shapiro-Wilk test de normalité} 
  \label{} 
\begin{tabular}{@{\extracolsep{5pt}} cccc} 
\\[-1.8ex]\hline 
\hline \\[-1.8ex] 
 & 2SLS & 3SLS & i3SLS \\ 
\hline \\[-1.8ex] 
Equation de demande & $0$ & $0$ & $0$ \\ 
Equation d'offre & $0$ & $0$ & $0$ \\ 
\hline \\[-1.8ex] 
\end{tabular} 
\end{table}

\FloatBarrier

\FloatBarrier

\begin{figure}[!htbp]

{\centering \includegraphics{note2pres_files/figure-latex/unnamed-chunk-91-1} 

}

\caption{Les PDF des résidus}\label{fig:unnamed-chunk-91}
\end{figure}

\FloatBarrier

\hypertarget{c5-comparaison-des-modeles}{%
\subsubsection{C5 Comparaison des
modèles}\label{c5-comparaison-des-modeles}}

\FloatBarrier

\FloatBarrier

\begin{table}[!htbp] \centering 
  \caption{Hausman 3SLS consistency test} 
  \label{} 
\begin{tabular}{@{\extracolsep{5pt}} ccc} 
\\[-1.8ex]\hline 
\hline \\[-1.8ex] 
 & Test & Resultats \\ 
\hline \\[-1.8ex] 
1 & 2SLS contre 3SLS & $0.827$ \\ 
2 & 2SLS contre i3SLS & $0.910$ \\ 
\hline \\[-1.8ex] 
\end{tabular} 
\end{table}

\FloatBarrier

\FloatBarrier

\FloatBarrier

\newpage

\hypertarget{d-clusterisation}{%
\subsection{D Clusterisation}\label{d-clusterisation}}

\hypertarget{d1-between-transformation}{%
\subsubsection{\texorpdfstring{D1 \emph{Between}
transformation}{D1 Between transformation}}\label{d1-between-transformation}}

Les groupes sont définies par des caractéristiques suivantes :

\FloatBarrier

\begin{table}[!htbp] \centering 
  \caption{Les centres des clusters} 
  \label{} 
\begin{tabular}{@{\extracolsep{5pt}} ccccccc} 
\\[-1.8ex]\hline 
\hline \\[-1.8ex] 
 & qi & ipi & si & ri & iki & .1 \\ 
\hline \\[-1.8ex] 
1 & $10.609$ & $1.393$ & $6.705$ & $9.880$ & $1.273$ & $27$ \\ 
2 & $7.468$ & $1.416$ & $4.345$ & $9.905$ & $1.238$ & $34$ \\ 
3 & $4.340$ & $1.622$ & $1.098$ & $9.870$ & $1.284$ & $8$ \\ 
\hline \\[-1.8ex] 
\end{tabular} 
\end{table}

\FloatBarrier

\hypertarget{d2-within-transformation}{%
\subsubsection{\texorpdfstring{D2 \emph{Within}
transformation}{D2 Within transformation}}\label{d2-within-transformation}}

\hypertarget{les-centres}{%
\paragraph{Les centres}\label{les-centres}}

Les groupes sont définies par des caractéristiques suivantes :

\FloatBarrier

\begin{table}[!htbp] \centering 
  \caption{Les centres des clusters} 
  \label{} 
\begin{tabular}{@{\extracolsep{5pt}} ccccccccc} 
\\[-1.8ex]\hline 
\hline \\[-1.8ex] 
 & qi & ipi & si & ri & iki & n & k & t \\ 
\hline \\[-1.8ex] 
1 & -0.252896 & -0.793219 & -0.07542 & -0.013193 & -0.256673 & 5 & 1 & 1 \\ 
2 & -0.362457 & -0.716949 & 0.041055 & -0.000747 & -0.036528 & 5 & 1 & 2 \\ 
3 & 0.166103 & 0.2392 & -0.013413 & -0.006388 & 0.157998 & 5 & 1 & 3 \\ 
4 & -0.215706 & -0.332061 & -0.199289 & 0.004172 & -0.019761 & 5 & 1 & 4 \\ 
5 & 0.664955 & 1.603029 & 0.247068 & 0.016155 & 0.154963 & 5 & 1 & 5 \\ 
6 & -0.173176 & -0.700641 & -0.063262 & -0.011699 & -0.091867 & 16 & 2 & 1 \\ 
7 & 0.281974 & 0.276242 & 0.030031 & 0.001569 & -0.073717 & 16 & 2 & 2 \\ 
8 & 0.318986 & 0.534754 & -0.017193 & -0.005653 & 0.076719 & 16 & 2 & 3 \\ 
9 & 0.272819 & 0.635118 & 0.096895 & 0.003199 & -0.147998 & 16 & 2 & 4 \\ 
10 & -0.700603 & -0.745473 & -0.046471 & 0.012585 & 0.236862 & 16 & 2 & 5 \\ 
11 & 0.053201 & -0.250299 & 0.089817 & -0.012167 & -0.261237 & 20 & 3 & 1 \\ 
12 & -0.150237 & -0.20443 & 0.168342 & -0.001309 & -0.012628 & 20 & 3 & 2 \\ 
13 & 0.061586 & 0.149078 & 0.01519 & -0.006477 & 0.094218 & 20 & 3 & 3 \\ 
14 & -0.023929 & 0.120274 & -0.129925 & 0.004456 & -0.074835 & 20 & 3 & 4 \\ 
15 & 0.059379 & 0.185377 & -0.143424 & 0.015498 & 0.254483 & 20 & 3 & 5 \\ 
16 & 0.126897 & -0.260636 & 0.047425 & -0.010084 & -0.134339 & 5 & 4 & 1 \\ 
17 & -0.719432 & -0.703363 & -0.254722 & 0.002955 & 0.095193 & 5 & 4 & 2 \\ 
18 & -0.141159 & -0.225144 & -0.158017 & -0.009434 & 0.29572 & 5 & 4 & 3 \\ 
19 & 0.176733 & 0.308934 & -0.1592 & 0.002576 & -0.259209 & 5 & 4 & 4 \\ 
20 & 0.556961 & 0.880209 & 0.524514 & 0.013987 & 0.002635 & 5 & 4 & 5 \\ 
21 & 0.282059 & -0.014017 & 0.400864 & -0.009269 & -0.409711 & 22 & 5 & 1 \\ 
22 & 0.204605 & 0.138187 & 0.285719 & 0.003743 & -0.134925 & 22 & 5 & 2 \\ 
23 & 0.274731 & 0.275854 & 0.118099 & -0.007893 & 0.166927 & 22 & 5 & 3 \\ 
24 & -0.058265 & 0.063243 & -0.18248 & 0.001702 & 0.015278 & 22 & 5 & 4 \\ 
25 & -0.70313 & -0.463267 & -0.622202 & 0.011716 & 0.362431 & 22 & 5 & 5 \\ 
26 & -1.595397 & -7.941881 & -0.261529 & -0.02149 & -0.049931 & 1 & 6 & 1 \\ 
27 & 0.253285 & -1.404496 & -0.153318 & 0.001641 & -0.013255 & 1 & 6 & 2 \\ 
28 & 0.529538 & 2.487263 & 0.747453 & -0.004844 & 0.080087 & 1 & 6 & 3 \\ 
29 & 0.936501 & 9.608734 & 0.226164 & 0.006494 & -0.034652 & 1 & 6 & 4 \\ 
30 & -0.123927 & -2.749621 & -0.55877 & 0.018199 & 0.017751 & 1 & 6 & 5 \\ 
\hline \\[-1.8ex] 
\end{tabular} 
\end{table}

\FloatBarrier

\hypertarget{representation-graphique}{%
\paragraph{Representation graphique}\label{representation-graphique}}

\begin{center}\includegraphics{note2pres_files/figure-latex/unnamed-chunk-99-1} \end{center}

\hypertarget{d3-cas-dinformation-complete}{%
\subsubsection{D3 Cas d'information
complete}\label{d3-cas-dinformation-complete}}

\hypertarget{les-centres-1}{%
\paragraph{Les centres}\label{les-centres-1}}

Les groupes sont définies par des caractéristiques suivantes :

\FloatBarrier

\begin{table}[!htbp] \centering 
  \caption{Les centres des clusters} 
  \label{} 
\begin{tabular}{@{\extracolsep{5pt}} ccccccccc} 
\\[-1.8ex]\hline 
\hline \\[-1.8ex] 
 & qi & ipi & si & ri & iki & n & k & t \\ 
\hline \\[-1.8ex] 
1 & 0.001186 & -0.359222 & 0.04895 & -0.012025 & -0.234518 & 29 & 1 & 1 \\ 
2 & -0.280464 & -0.376425 & 0.078291 & -0.000723 & 0.004804 & 29 & 1 & 2 \\ 
3 & 0.049743 & 0.107731 & -0.01841 & -0.006913 & 0.122468 & 29 & 1 & 3 \\ 
4 & -0.032737 & 0.065232 & -0.147634 & 0.004155 & -0.089222 & 29 & 1 & 4 \\ 
5 & 0.262272 & 0.562683 & 0.038803 & 0.015506 & 0.196468 & 29 & 1 & 5 \\ 
6 & 0.098311 & -0.291704 & 0.206247 & -0.01031 & -0.27852 & 39 & 2 & 1 \\ 
7 & 0.223902 & 0.18426 & 0.174215 & 0.002905 & -0.108882 & 39 & 2 & 2 \\ 
8 & 0.283634 & 0.373141 & 0.059067 & -0.006982 & 0.141058 & 39 & 2 & 3 \\ 
9 & 0.086134 & 0.306445 & -0.065995 & 0.002333 & -0.059896 & 39 & 2 & 4 \\ 
10 & -0.691981 & -0.572142 & -0.373534 & 0.012054 & 0.30624 & 39 & 2 & 5 \\ 
11 & -1.595397 & -7.941881 & -0.261529 & -0.02149 & -0.049931 & 1 & 3 & 1 \\ 
12 & 0.253285 & -1.404496 & -0.153318 & 0.001641 & -0.013255 & 1 & 3 & 2 \\ 
13 & 0.529538 & 2.487263 & 0.747453 & -0.004844 & 0.080087 & 1 & 3 & 3 \\ 
14 & 0.936501 & 9.608734 & 0.226164 & 0.006494 & -0.034652 & 1 & 3 & 4 \\ 
15 & -0.123927 & -2.749621 & -0.55877 & 0.018199 & 0.017751 & 1 & 3 & 5 \\ 
\hline \\[-1.8ex] 
\end{tabular} 
\end{table}

\FloatBarrier

\hypertarget{representation-graphique-1}{%
\paragraph{Representation graphique}\label{representation-graphique-1}}

\FloatBarrier

\begin{center}\includegraphics{note2pres_files/figure-latex/unnamed-chunk-102-1} \end{center}

\FloatBarrier

\hypertarget{e-analyse-des-resultats-pour-information-clusterisee-ols-2sls-et-3sls}{%
\subsection{E Analyse des résultats pour information clusterisée OLS,
2SLS et
3SLS}\label{e-analyse-des-resultats-pour-information-clusterisee-ols-2sls-et-3sls}}

Etudions la validité du modèle 3SLS :

\FloatBarrier

\FloatBarrier

\begin{table}[!htbp] \centering 
  \caption{Hausman 3SLS consistency test} 
  \label{} 
\begin{tabular}{@{\extracolsep{5pt}} ccc} 
\\[-1.8ex]\hline 
\hline \\[-1.8ex] 
 & Test & Resultats \\ 
\hline \\[-1.8ex] 
1 & 2SLS contre 3SLS & $0$ \\ 
\hline \\[-1.8ex] 
\end{tabular} 
\end{table}

La normalité des résidus :

\FloatBarrier

\begin{table}[!htbp] \centering 
  \caption{Shapiro-Wilk normality test} 
  \label{} 
\begin{tabular}{@{\extracolsep{5pt}} cccc} 
\\[-1.8ex]\hline 
\hline \\[-1.8ex] 
 & OLS & 2SLS & 3SLS \\ 
\hline \\[-1.8ex] 
Equation de demande & $0$ & $0.00000$ & $0$ \\ 
Equation d'offre & $0$ & $0$ & $0$ \\ 
\hline \\[-1.8ex] 
\end{tabular} 
\end{table}

\FloatBarrier

L'heteroscedacité :

\FloatBarrier

\FloatBarrier

\begin{table}[!htbp] \centering 
  \caption{Bartlett heteroscedasticity test} 
  \label{} 
\begin{tabular}{@{\extracolsep{5pt}} cccc} 
\\[-1.8ex]\hline 
\hline \\[-1.8ex] 
 & OLS & 2SLS & 3SLS \\ 
\hline \\[-1.8ex] 
Equation de demande & $0$ & $0.00002$ & $0.00000$ \\ 
Equation d'offre & $0$ & $0.00000$ & $0.00000$ \\ 
\hline \\[-1.8ex] 
\end{tabular} 
\end{table}

\FloatBarrier

Les PDF des résidus :

\FloatBarrier

\begin{center}\includegraphics{note2pres_files/figure-latex/unnamed-chunk-110-1} \end{center}

\FloatBarrier

Les résidus contre les variables prédites :

\FloatBarrier

\begin{center}\includegraphics{note2pres_files/figure-latex/unnamed-chunk-111-1} \end{center}

\FloatBarrier

\newpage

\hypertarget{f-dictionnaire-des-variables}{%
\subsection{F Dictionnaire des
variables}\label{f-dictionnaire-des-variables}}

Finalement, nous offrons au lecteur un tableau de reference pour notre
base des données finale.

\FloatBarrier

\begin{table}[!htbp]
  \centering
\caption{Ditionnaire des varibales}
\begin{tabular}{c|l}
  \noindent\rule[0.5ex]{\linewidth}{1pt}
  Variable & Description \\
  \noindent\rule[0.5ex]{\linewidth}{1pt}
année & année \\
ndep & numéro de département \\
si & superficie de vigne sans indication géographique en hectare en log \\
qi & quantité de vins produits en hectolitre en log \\
ipi & indice des prix du vin sans IG déflatés en log  \\
ri & revenu disponible brut des ménages français déflatés en log \\
iki & indice de quantité de pesticides achetés en log \\
t & la tendance temporelle \\
\noindent\rule[0.5ex]{\linewidth}{1pt}
\end{tabular}
\end{table}

\FloatBarrier

\newpage

\hypertarget{references}{%
\section*{References}\label{references}}
\addcontentsline{toc}{section}{References}

\hypertarget{refs}{}
\leavevmode\hypertarget{ref-anderson2011global}{}%
Anderson, Kym, Signe Nelgen, and others. 2011. \emph{Global Wine
Markets, 1961 to 2009: A Statistical Compendium}. University of Adelaide
Press.

\leavevmode\hypertarget{ref-cembalo2014}{}%
Cembalo, Luigi, Francesco Caracciolo, and Eugenio Pomarici. 2014.
``Drinking Cheaply: The Demand for Basic Wine in Italy.''
\emph{Australian Journal of Agricultural and Resource Economics} 58 (3):
374--91.

\leavevmode\hypertarget{ref-kremer2004}{}%
KREMER, Florence, and Catherine VIOT. 2004. ``Conflit et Coopération Au
Sein Du Canal: L'interaction Stratégique Entre La Grande Distribution et
Les Producteurs de La Filière Viti-Vinicole.''

\leavevmode\hypertarget{ref-laporte1996}{}%
Laporte, Catherine, and Marie-Claude PICHERY. 1996. ``Production costs
of AOC Burgundy wines.'' Research Report. Laboratoire d'analyse et de
techniques économiques(LATEC).
\url{https://hal.archives-ouvertes.fr/hal-01526958}.

\leavevmode\hypertarget{ref-mackay2018}{}%
MacKay, Alexander, and Nathan H Miller. 2018. ``Estimating Models of
Supply and Demand: Instruments and Covariance Restrictions.''

\leavevmode\hypertarget{ref-makela2006}{}%
MÄKELÄ, PIA, GERHARD GMEL, ULRIKE GRITTNER, HERVÉ KUENDIG, SANDRA
KUNTSCHE, KIM BLOOMFIELD, and ROBIN ROOM. 2006. ``DRINKING PATTERNS AND
THEIR GENDER DIFFERENCES IN EUROPE.'' \emph{Alcohol and Alcoholism} 41
(October): i8--i18. \url{https://doi.org/10.1093/alcalc/agl071}.

\leavevmode\hypertarget{ref-outreville2010}{}%
Outreville, J François. 2010. ``Les Facteurs Déterminant Le Prix Du
Vin.'' \emph{Enometrica} 3 (1): 25--33.

\leavevmode\hypertarget{ref-steiner2004}{}%
Steiner, Bodo. 2004. ``French Wines on the Decline? Econometric Evidence
from Britain.'' \emph{Journal of Agricultural Economics} 55 (2):
267--88.


\end{document}
