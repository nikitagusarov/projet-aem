\documentclass[11pt, a4paper]{article}

\usepackage[utf8]{inputenc}
\usepackage[french]{babel}
\AddThinSpaceBeforeFootnotes
\FrenchFootnotes
\usepackage[T1]{fontenc}
\usepackage{amsmath}
\usepackage{booktabs}
\usepackage{longtable}
\usepackage{amsfonts}
\usepackage{amssymb, amsthm}
\usepackage{supertabular}
\usepackage[dvipsnames]{xcolor}
\usepackage{geometry}
\usepackage{array}
\setlength{\parindent}{0.1em}
%\setcellgapes{0.1pt}
%\makegapedcells
\newcolumntype{R}[1]{>{\raggedleft\arraybackslash }b{#1}}
\newcolumntype{L}[1]{>{\raggedright\arraybackslash}b{#1}}
\newcolumntype{C}[1]{>{\centering\arraybackslash}b{#1}}
\geometry{hmargin=2.5cm, vmargin=2.5cm}
%\DeclareUnicodeCharacter{00A0}{ }
\setlength{\parskip}{.5em}
\usepackage{comment}
\usepackage[graphicx]{realboxes}
\usepackage[section]{placeins}
\usepackage{adjustbox}
\usepackage{etex}
\usepackage[hypertexnames=false]{hyperref}
\hypersetup{colorlinks = true, linkcolor = Black, urlcolor = Blue, citecolor = Black}
\usepackage{tabu}
\usepackage{pgf, tikz}
\usetikzlibrary{positioning,chains,fit,shapes,calc}
\usetikzlibrary{arrows}
\usepackage{bm}
%\usepackage{lscape}
\usepackage{pdflscape}
\usepackage[splitrule]{footmisc}
\usepackage[nottoc, notlof, notlot]{tocbibind}
\usepackage[clockwise]{rotating}
\renewcommand{\baselinestretch}{1}
%\renewcommand{\thesection}{\Roman{section}}
\usepackage{titlesec}
\titlespacing\section{0pt}{12pt plus 4pt minus 2pt}{0pt plus 2pt minus 2pt}
\titlespacing\subsection{0pt}{12pt plus 4pt minus 2pt}{0pt plus 2pt minus 2pt}
\titlespacing\subsubsection{0pt}{12pt plus 4pt minus 2pt}{0pt plus 2pt minus 2pt}
\usepackage{cite}
\usepackage{easytable}
\usepackage{caption}
\usepackage{lipsum}
\usepackage{alltt}
\usepackage{graphicx}
\usepackage[authoryear, round]{natbib}
%\usepackage{fontspec}
%\usepackage[french]{babel}
%\usepackage{fontspec}
%\usepackage{xunicode}
%\usepackage{xltxtra}
%\setmainfont{Calibri}
\pagenumbering{arabic}
\addto\captionsfrench{\def\tablename{Tableau}}
\DeclareMathOperator*{\argmin}{argmin}
\DeclareMathOperator*{\argmax}{argmax}
%\usepackage{usebib}
%\newcommand{\printarticle}[1]{\citeauthor{#1}, ``\usebibentry{#1}{title}''}
%\bibinput{biblio}
%\usepackage{bibtex}
\bibliographystyle{plainnat}

\begin{document}

\begin{center}
    \Large\textbf{État d'avancement préliminaire}
    \par
    \large\textit{A. Blanc, N. Gusarov, S. Picon}
\end{center}

% Merci de faire une note avec votre état d'avancement sur la réalisation de votre projet empirique.
% Afin de rendre ce rendu bénéfique : merci d'identifier précisément les points de blocage.
% Je vous demande de me renvoyer ces éléments sur Modèle avant le 5 novembre à 17h00 que je puisse les consulter avant notre cours.

\section*{Question économique traitée}
Dans notre travail nous étudions les impacts des fluctuations du prix des vins de table (les vins simples) sur la demande des pesticides par des agriculteurs français.

\section*{Références et synthèses}
% référence et synthèse des 2/3 articles fondateurs
Pour le moment nous avons étudié et synthétisé 14 articles sur le marché du vin, ainsi bien que 2 articles sur le traitement économétrique des systèmes d'équations simultanés. 
\par
Parmi les articles traitant la viticulture et le marché du vin, nous avons les travaux de :
\begin{itemize}
    \item \citet*{rebelo2019}
    \item \citet*{anderson2018s}
    \item \citet*{bazen2018}
    \item \citet*{garcia2016}
    \item \citet*{cembalo2014}
    \item \citet*{outreville2010}
    \item \citet*{benfratello2009}
    \item \citet*{davis2008}
    \item \citet*{christophe2008}
    \item \citet*{costanigro2007}
    \item \citet*{steiner2004}
    \item \citet*{soler2002}
    \item \citet*{giraud2001}
    \item \citet*{gergaud1998}
    \item \citet*{labys1976}
    \item \citet*{daillens1962}
\end{itemize}
Les travaux théoriques et méthodologiques sont des extraits du manuel de Wooldridge :
\begin{itemize}
    \item \citet*{wooldridge2016introductory}
    \item \citet*{wooldridgepanel}
\end{itemize}
\par
Maintenant, nous allons donner un aperçu des articles les plus pertinents dans notre cas d'étude.

\subsection*{"Drinking cheaply: the demand for basic wine in Italy"}
Un travail de Luigi Cembalo, Francesco Caracciolo et Eugenio Pomarici, publié en 2014.
Ce recherche nous servira pour la base, car il traite la plus part des problèmes dont on rencontre dans cet étude. 
\par
Cet article détermine s'il existence un éventuel degré d’hétérogénéité dans les vins non premium et essaye le mesurer.
Cela est fait par calcul de l’élasticité, les relations entre les catégories de vins agrégées avec des critères allant au-delà du prix (producteur, type, couleur, etc). 
Un système de demande (modèle QUAIDS) a été estimé à l'aide d'un panel statistiquement représentatif de 6 773 ménages italiens, afin de déterminer dans quelle mesure une éventuelle substitution se produit dans la consommation domestique de vins de base (qui constitue le principal canal de distribution des vins bon marché en Italie).
\par
Les résultats les plus saillants sont la faible élasticité du vin en carton du leader du marché (ML) et la grande élasticité du vin non-marque en carton (OC), ce qui indique sur la présence d'un degré d'hétérogénéité. 
Sur la base des résultats, les auteurs affirment que, même dans le segment des vins plus économiques, la marque est un instrument efficace de diversification.
\par
Nous dans notre analyse, supposons, que ces effets d'hétérogénéité sont similaires pour des groupes des marques, agrégées par région.

\subsection*{"French Wines on the decline ? Econometric evidence from Britain"}
Cet analyse effectué par Bodo Steiner en 2002 contribue à la littérature existante des prix hédoniques sur le marché du vin en UK. 
\par
Les auteurs développent l’approche des variables nominales % mise au point par Kennedy (1986) et Oczkowski (1994) 
pour obtenir une contribution distincte et comparable de chaque attribut à la variation des prix. 
Dans leurs étude, % contrairement à Combris et al. (1997), 
ils s'appuis pas sur des caractéristiques sensorielles, mais sur des attributs que les consommateurs peuvent observer via l'étiquette (cépage, millésime, etc.). 
De plus, contrairement aux études hédoniques précédentes sur le vin, les chercheurs n'utilisent pas les prix de détail recommandés, mais les prix de détail réels.
\par
L’approche économétrique utilisé est un estimateur des moindres carrés avec la correction pour l’hétéroscédasticité. 
\par
Les estimations suggèrent que les vins de cépages français, tels qu’ils sont apparus dans les points de vente au détail en 1994, n’avaient pas obtenu une évaluation suffisamment positive des acteurs du marché britannique. 
Les auteurs constatent la diminution globale du rôle des vins français en Grande-Bretagne.

\subsection*{"Les facteurs déterminants les prix du vin"}
Un travail de J. François Outreville fait en 2009.
\par
L'auteur identifie les facteurs les facteurs les plus fréquemment utilisé dans la littérature portant sur l'analyse des vins et du marché du vin.
\begin{center}
\begin{tabular}{l|l|l}
    \hline
    \hline
    Facteurs géo-viticoles & Facteurs temporels & Coûts de production \\
    \hline
    ~~~~ Climat & ~~~~Age & ~~~~Coûts fixes et variables \\
    ~~~~ Sol & ~~~~Millésime & ~~~~Type de vin \\
    ~~~~ Région & & ~~~~Rendement \\
    ~~~~ Cépage & & \\
    \hline
\end{tabular}
\par
\begin{tabular}{l|l}
    \hline
    \hline
    Information & Offre et demande \\
    \hline
    ~~~~Etiquette & ~~~~Rareté \\
    ~~~~Appellation & ~~~~Culture \\
    ~~~~Réputation & \\
    & \\
    \hline
\end{tabular}
\end{center}

\subsection*{"Pesticide regulation : the case of French wine"}
C'est article est un "\textit{working paper}" rédigé par Deola Cristophe et Fleckinger Pierre en 2008.
\par
Les auteurs identifient que le régime d’appelation favorise l’utilisation des pesticides tout en limitant la quantité de raisins utilisée dans la production de vin par rapport à la situation non réglementée. 
\par
Consécutivement, les rendements des plantation viticoles sont limitées par certaines réglementation professionnelles si bien que l’utilisation des pesticides est peu sensible à un faible niveau d’imposition car incitation à administrer une plus forte dose de pesticides sur les cultures pour permettre aux producteurs de garantir une bonne qualité du vin produit et donc la perception de revenus suffisants pour que leur exploitation perpétue dans le temps. 
Une réglementation économique à elle seule n’est pas suffisante pour limiter les dommages environnementaux commis par les pesticides.
\par
Les résultats de cet article nous permet de supposer que une faible élasticité-prix de la demande des pesticides nous permet ignorer les relations agriculteur-producteur du marché des pesticides.
Nous supposons, que la demande des pesticides ne depend que des besoins des agriculteurs.

\section*{Modèle économique}
% Cembalo2014
Dans le commerce du vin, il est courant de diviser les vins en deux grandes classes en fonction de leurs prix \cite{cembalo2014} : 
\begin{itemize}
    \item les vins de qualité inférieure, les moins chers avec les caractéristiques de qualité de base ;
    \item les vins de qualité supérieure plus chers, dotés de caractéristiques qualitatives complexes et d'une image de grande valeur.
\end{itemize} % (Anderson et Nelgen 2011).
\par
De plus, pour les vins français, selon \citet{steiner2004}, le système européen de classification des "\textit{vins de qualité produits dans certaines régions}" (VQPRD) contient à la fois des vins AOC et des "\textit{vins de haute qualité provenant d'un vignoble régional agréé}" (VDQS). 
Les vins de cépage appartiennent à la catégorie des vins autres que VQPRD, qui comprend les \textbf{vins de table} et les \textbf{vins de pays}.
\par
Tenant en compte cet information, nous utilisons la méthodologie du ministère d'agriculture et divisons le marché en deux parties :
\begin{itemize}
    \item La haute gamme (les vins IGP, vendus dans des magasins spécifiques) ;
    \item La gamme basse (les vins non IGP, vendus en grands surfaces).
\end{itemize}
\par
La première partie est soumise à des règlements spécifiques : limitations des quantités produites, origine contrôlé, un caractère de la demande spécifique. 
La deuxième, c'est-à-dire le marché des vins moins chers, est aussi en quelque sens complexe. Les produits, classés dans cette catégorie proposée, sont susceptible d'avoir un certain degré d'hétérogénéité, comme cela a été montré par \citet{cembalo2014}.
\par
Nous, dans notre étude, traitons que les vins simples (non IGP). 
La situation sur cet marché est sensé d'influencer le plus l'utilisation des pesticides, car les volumes de productions sont plus significatives que pour le marché des vins IGP.
\par 
Suivant le raisonnement des chercheurs \cite{cembalo2014}, dans une catégorie de vin avec une fourchette de prix étroite, il existe une homogénéité presque parfaite due à des vins ayant des attributs intrinsèques simples, une complexité de qualité médiocre et donc une différenciation peu marquée.
Nous ignorons les interactions internationales. 
Cela nous permet d'analyser le marché par régions est non par des marques/produits.
\par
Comme proposé dans la littérature, notre étude sur les vins non coûteux (non IGP) est effectué au niveau du pays \cite{cembalo2014} pour deux raisons :
\begin{itemize}
    \item Les prix de vente moyens des marchés diffèrent en raison des droits de douane à l'importation et des taxes à la consommation différents % (Anderson et Nelgen 2011);
    \item La perception des produits de consommation varie d'un pays à l'autre % (Makela et al. 2006).
\end{itemize}
Quand aux exportations et les importations, n'ayant pas la possibilité contrôler le montant des vins non IGP exportés/importés, nous laissons ces effets au terme d'erreur. 
\par
Pour conclure, nos suppositions au niveau du marché de vin sont suivantes :
\begin{itemize}
    \item La demande pour les vins simples est unique pour toute la France. On n'observe pas les quantités consommés par régions, mais pour tout le pays, avec un prix unique. 
    \item La production du vin varie par région, suite à des différences climatologiques.
    \item On n'observe que l'équilibre sur le marché au niveau du pays (la quantité demandé est égale à la quantité offerte par l'ensemble des régions).
\end{itemize}
\par
En ce qui concerne les pesticides, nous supposons que :
\begin{itemize}
    \item La demande des pesticides est inélastique par prix, ce qui nous permet d'exclure la partie de l'offre des pesticides du notre analyse. La quantité des pesticides utilisée demande seulement des intentions et des besoins des agriculteurs. 
    \item 
\end{itemize}
\par
% Cembalo2014
% A Quadratic Almost Ideal Demand System (QUAIDS) was implemented (Banks et al. 1997). The use of a model allowing a more general Engel curve shape than the popular Almost Ideal Demand System (AIDS) of Deaton and Muellbauer (1980) was required.
% Banks et al. (1997) show that the demand for some goods, particularly alcohol and clothing, has a quadratic relationship with the logarithm of total expenditure at higher income levels.
Formalisant notre modèle théorique, nous posons, que la demande de vin a la forme suivante :
\begin{equation}
    Q_d = \alpha_d + \beta_d P_d + \gamma_d Z 
\end{equation}
Avec $Z$ étant l'ensemble des variables ayant l'influence sur la demande du vin, dans le cas le plus simple nous n'utilisons que les revenus (c'est une des variables les plus utilisées dans des études empiriques sur le marché du vin).
\par
L'offre totale pour toute la France est donnée par l'équation suivante : 
\begin{equation}
    Q_o = \sum_{i = 1}^{N} q_i
\end{equation}
Ou $i \in \{1, ..., N\}$ sont des régions, chacun ayant sa propre fonction de production et d'offre unique : 
\begin{equation}
    q_i = a_i + b_i P_o + c_i X
\end{equation}
Avec $X$ étant un vecteur des variables explicatives influençant la production (dans le cas le plus simple nous ne prenons en compte que les quantités des pesticides utilisées).
Nous pouvons réécrire l'équation de l'offre sous la forme :
\begin{equation}
    Q_o = \sum_{i = 1}^{N} (a_i + b_i P_o + c_i X) = \sum_{i = 1}^{N} a_i + \sum_{i = 1}^{N} b_i P_o + \sum_{i = 1}^{N} c_i X
\end{equation}
Nous obtenons enfin un système de $N + 2$ équations : 
\begin{align*}
    Q_d & = \alpha_d + \beta_d P_d + \gamma_d Z \\
    Q_o & = \sum_{i = 1}^{N} q_i \\
    q_1 & = a_1 + b_1 P_o + c_1 X \\ 
    \vdots \\ 
    q_N & = a_N + b_N P_o + c_N X \\
\end{align*}
Théoriquement, nous pouvons aussi supposer la presence des prix marginaux non constants et la constance des prix relatifs par rapport aux modifications des proportions des caractéristiques, comme cela a été fait par \citet{steiner2004}. 
Cette spécification suppose une fonction log-linéaire. 
Cela nous permettra de passer à l'échelle logarithmique, ce qui peut être avantageux (bien que cela risque de poser des problèmes lors d'estimation).

\section*{Les données}
Nous avons utilisé les bases des données suivantes pour notre analyse :
\begin{itemize}
    \item AGRESTE (les données sur les pesticides)
    \item France AgriMer (les prix du vin)
    \item INSEE (les données sur la population)
    \item Les données statistiques du ministère des finances (les données sur le vin)
\end{itemize}

\subsection*{Les données mobilisées}
% les étapes de la construction de votre BDD ainsi que les différents appariements sont à mentionner.


\subsection*{Statistiques descriptives des variables clé}
% + analyses bivariées avec la variable dépendante.

\section*{Modèle économétrique}
L'AIDS et les autres modèles de demande cités dans la littérature ont de nombreuses lacunes qui les rendent impropres pour l'estimation du marché du vin, selon \citet{cembalo2014}. 
Quand même, dans notre étude nous allons utiliser ce modèle là, sous des suppositions restrictives. 
\par
Comme c'était déjà mentionné dans la partie théorique, nous n'observons que les quantités et les prix moyens pour les points des équilibres où $Q_o = Q_d = Q$ et $P_o = P_d = P$.
Ce fait nous amène à l'utilisation des systèmes des équations simultanées, ce qui nous permettra de capter les décisions simultanées des vendeurs et des acheteurs. 
\par 
Afin de pouvoir estimer le système, qu'on a spécifié dans la partie précédente, nous allons d'abord nos servir de la suppositions que au point d'équilibre observé nous avons : $Q_o = Q_d = Q$.
Alors, pour l'offre régionale :
\begin{equation}
    q_i = a_i + b_i P_o + c_i X + u_i
\end{equation} 
On a :
\begin{equation}
    \alpha_d + \beta_d P + \gamma_d Z + \epsilon = 
        \sum_{i = 1}^{N} a_i + \sum_{i = 1}^{N} b_i P + \sum_{i = 1}^{N} c_i X + \sum_{i = 1}^{N} u_i
\end{equation}
D'où on obtient :
\begin{equation}
    P = \frac{\sum_{i = 1}^{N} a_i - \alpha_d}{\beta_d - \sum_{i = 1}^{N} b_i} + 
        \frac{\sum_{i = 1}^{N} c_i}{\beta_d - \sum_{i = 1}^{N} b_i} X +
        \frac{-\gamma_d}{\beta_d - \sum_{i = 1}^{N} b_i} Z + 
        \frac{\sum_{i = 1}^{N} - \epsilon}{\beta_d - \sum_{i = 1}^{N} b_i}
\end{equation}
Respectivement on peut dériver équation structurelle pour $Q$ :
\begin{multline}
    Q = (\alpha_d + \beta_d \frac{\sum_{i = 1}^{N} a_i - \alpha_d}{\beta_d - \sum_{i = 1}^{N} b_i}) + 
        (\beta_d \frac{\sum_{i = 1}^{N} c_i}{\beta_d - \sum_{i = 1}^{N} b_i}) X + \\
        (\gamma_d + \beta_d \frac{-\gamma_d}{\beta_d - \sum_{i = 1}^{N} b_i}) Z + 
        (\epsilon + \beta_d \frac{\sum_{i = 1}^{N} - \epsilon}{\beta_d - \sum_{i = 1}^{N} b_i})
\end{multline}

\subsection*{Méthode d'estimation envisagée}
Nous devrions estimer les les équations structurelles identifiés dans la partie précédente de cette note. 
Ayant identifié ces coefficients nous pourrions en dériver les coefficients des équations de départ, dont on a besoin pour faire des conclusions.

% \subsection*{Premiers résultats économétriques} 

\bibliography{biblio}

\end{document}