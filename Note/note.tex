\documentclass[11pt, a4paper]{article}

\usepackage[utf8]{inputenc}
\usepackage[french]{babel}
\AddThinSpaceBeforeFootnotes
\FrenchFootnotes
\usepackage[T1]{fontenc}
\usepackage{amsmath}
\usepackage{booktabs}
\usepackage{longtable}
\usepackage{amsfonts}
\usepackage{amssymb, amsthm}
\usepackage{supertabular}
\usepackage[dvipsnames]{xcolor}
\usepackage{geometry}
\usepackage{array}
\setlength{\parindent}{0.1em}
%\setcellgapes{0.1pt}
%\makegapedcells
\newcolumntype{R}[1]{>{\raggedleft\arraybackslash }b{#1}}
\newcolumntype{L}[1]{>{\raggedright\arraybackslash}b{#1}}
\newcolumntype{C}[1]{>{\centering\arraybackslash}b{#1}}
\geometry{hmargin=2.5cm, vmargin=2.5cm}
%\DeclareUnicodeCharacter{00A0}{ }
\setlength{\parskip}{.5em}
\usepackage{comment}
\usepackage[graphicx]{realboxes}
\usepackage[section]{placeins}
\usepackage{adjustbox}
\usepackage{etex}
\usepackage[hypertexnames=false]{hyperref}
\hypersetup{colorlinks = true, linkcolor = Black, urlcolor = Blue, citecolor = Black}
\usepackage{tabu}
\usepackage{pgf, tikz}
\usetikzlibrary{positioning,chains,fit,shapes,calc}
\usetikzlibrary{arrows}
\usepackage{bm}
%\usepackage{lscape}
\usepackage{pdflscape}
\usepackage[splitrule]{footmisc}
\usepackage[nottoc, notlof, notlot]{tocbibind}
\usepackage[clockwise]{rotating}
\renewcommand{\baselinestretch}{1}
%\renewcommand{\thesection}{\Roman{section}}
\usepackage{titlesec}
\titlespacing\section{0pt}{12pt plus 4pt minus 2pt}{0pt plus 2pt minus 2pt}
\titlespacing\subsection{0pt}{12pt plus 4pt minus 2pt}{0pt plus 2pt minus 2pt}
\titlespacing\subsubsection{0pt}{12pt plus 4pt minus 2pt}{0pt plus 2pt minus 2pt}
\usepackage{cite}
\usepackage{easytable}
\usepackage{caption}
\usepackage{lipsum}
\usepackage{alltt}
\usepackage{graphicx}
%\usepackage{fontspec}
%\usepackage[french]{babel}
%\usepackage{fontspec}
%\usepackage{xunicode}
%\usepackage{xltxtra}
%\setmainfont{Calibri}
\pagenumbering{arabic}
\addto\captionsfrench{\def\tablename{Tableau}}
\DeclareMathOperator*{\argmin}{argmin}
\DeclareMathOperator*{\argmax}{argmax}

\begin{document}

\begin{center}
    \Large\textbf{État d'avancement préliminaire}
    \par
    \large\textit{A. Blanc, N. Gusarov, S. Picon}
\end{center}

% Merci de faire une note avec votre état d'avancement sur la réalisation de votre projet empirique.
% Afin de rendre ce rendu bénéfique : merci d'identifier précisément les points de blocage.
% Je vous demande de me renvoyer ces éléments sur Modèle avant le 5 novembre à 17h00 que je puisse les consulter avant notre cours.

\section*{Question économique traitée}

\section*{Références et synthèses}
% référence et synthèse des 2/3 articles fondateurs

\section*{Modèle économique}

\section*{Les données}

\subsection*{Les données mobilisées}
% les étapes de la construction de votre BDD ainsi que les différents appariements sont à mentionner.

\subsection*{Statistiques descriptives des variables clé}
% + analyses bivariées avec la variable dépendante.

\section*{Modèle économétrique}

\subsection*{Méthode d'estimation envisagée}

% \subsection*{Premiers résultats économétriques} 

\end{document}